\documentclass{article}
\usepackage[utf8]{inputenc}

\usepackage{amsmath}
\usepackage{amsfonts}
\usepackage{bbm}
\usepackage{amsmath, amssymb, amsfonts}

\usepackage{graphicx}
\usepackage{subfigure}
\usepackage{color}
%\usepackage[round]{natbib}
\usepackage[hyperindex,breaklinks]{hyperref}
\usepackage{xspace}
\usepackage{framed}
\usepackage{algorithm}
\usepackage[noend]{algpseudocode}
%\usepackage{lipsum}
\usepackage{bm}
\usepackage{bbm}
\usepackage{mathtools}
\usepackage{todonotes}
%\usepackage{ulem}
\usepackage{caption}
%\usepackage{subcaption}
\usepackage{algpseudocode}
\usepackage[normalem]{ulem}
%defines page style
\usepackage[a4paper]{geometry} \geometry{top=1in, bottom=1in,
  left=1in, right=1in}
\setlength{\parskip}{0.5em}
\let\proof\relax
\let\endproof\relax
\let\example\relax
\let\endexample\relax

\usepackage{amsthm}
\newtheorem{theorem}{Theorem}[section]
\newtheorem{acknowledgement}[theorem]{Acknowledgement}
\newtheorem{assumption}{Assumption}
\newtheorem{case}{Case}
%\newtheorem{claim}{Claim}
\newtheorem{conclusion}{Conclusion}
\newtheorem{condition}[theorem]{Condition}
\newtheorem{conjecture}[theorem]{Conjecture}
\newtheorem{corollary}{Corollary}[section]
\newtheorem{criterion}{Criterion}
\newtheorem{definition}{Definition}
\newtheorem{lemma}{Lemma}[section]
\newtheorem{notation}[theorem]{Notational Convention} \newtheorem{proposition}{Proposition}[section]
\newtheorem{remark}{Remark}
\newtheorem{algo}{Algorithm}[section]
\theoremstyle{definition}
\newtheorem{example}{Example}
%\newtheorem{proof}{Proof}

%\theoremstyle{remark}
%\newtheorem{ex}{Example}

\hypersetup{colorlinks = true, urlcolor = blue, linkcolor = blue,
  citecolor = red }

\DeclareMathOperator*{\argmin}{arg\,min}

\newcommand{\xmath}[1]{\ensuremath{#1}\xspace}
\newcommand{\mref}{\xmath{\mu}}
\renewcommand{\Pr}{\xmath{P}}
\newcommand{\bbeta}{\xmath{\boldsymbol{\theta}}}
\newcommand{\ggamma}{\xmath{\boldsymbol{\gamma}}}
\newcommand{\xx}{\mathbf{x}}
\newcommand{\yy}{\mathbf{y}}
\newcommand{\ssigma}{\xmath{\boldsymbol{\sigma}}}
\newcommand{\ddelta}{\xmath{\boldsymbol{\delta}}}
\newcommand{\pphi}{\xmath{\boldsymbol{\phi}}}
\newcommand{\derCvarlow}{\nabla C_{t}(\bbeta)}
\newcommand{\derCvarHigh}{\nabla C_{t^{'}}(\bbeta)}
\newcommand{\derCvarlowN}{\nabla C_{t_n}(\bbeta)}
\newcommand{\derCvarHighN}{\nabla C_{t^{'}_n}(\bbeta)}
\newcommand{\derCvarapprox}{\hat\nabla_{n} C_{t}(\bbeta)}
\newcommand{\Real}{\mathbb{R}}
\newcommand{\Prob}{{P}}
\newcommand{\Expc}{\mathbb{E}}
\newcommand{\VarT}{\widetilde{\mathrm{Var}}}
\newcommand{\cis}{\hat{C}_{\beta,n}^{\textnormal{IS}}}
\newcommand{\vis}{\hat{v}_{\beta,n}^{\textnormal{IS}}}
\newcommand{\Fis}{\hat{F}_{\bbeta,n}^{\textnormal{IS}}}
\newcommand{\vsa}{\hat{v}_{\beta_0,n}}%^{\textnormal{SA}}}
\newcommand{\csa}{\hat{C}_{\beta_0,n}}%^{\textnormal{SA}}}

\newcommand{\lrset}[1]{\left\{{#1}\right\}}
\newcommand{\lrp}[1]{\left({#1}\right)}
\newcommand{\abs}[1]{\left\lvert{#1}\right\rvert}
\newcommand\numberthis{\addtocounter{equation}{1}\tag{\theequation}}
\newcommand{\Exp}[1]{\mathbb{E}\lrp{#1}}
\newcommand{\E}[2]{\mathbb{E}_{#1}\lrp{#2}}
\newcommand{\inv}{^{\text{-}1}}


\DeclareOldFontCommand{\bf}{\normalfont\bfseries}{\mathbf}

\title{Agent based simulators for Covid-19: Simulating larger models using smaller ones
}

\author{Daksh Mittal,  Sandeep Juneja,   Shubhada Agrawal  \\
\vspace{0.1in}
TIFR Mumbai}

\begin{document}


\maketitle


\begin{abstract}

Agent-based simulators (ABS) are a popular epidemiological modelling tool to study the impact of various  non-pharmaceutical interventions in managing an epidemic in a city (or a region). 
They provide the flexibility to accurately model a heterogeneous population with time and location varying, person-specific interactions. Government policies such as partial and localised lock downs, case isolation, home quarantine, school closures, partially opened workplaces, etc.  and important pandemic developments over time including presence of variants as well as vaccines, are easily incorporated in an ABS. Typically, for accuracy, each person is modelled separately. This however may make computational time prohibitive when the city population  and the simulated time is large.  In this paper, we primarily focus on the COVID-19 pandemic and dig deeper into the underlying probabilistic structure of a generic ABS to arrive at modifications that allow smaller models to give accurate statistics for larger ones. We observe that simply considering a smaller aggregate model and scaling up the output leads to inaccuracies. We exploit the observation that  in the initial Covid spread phase, the starting infections create a family tree of infected individuals more-or-less independent of the other trees and are modelled well as a multi-type super-critical branching process. Further, although this branching process grows exponentially, the relative proportions amongst the population  types stabilise  quickly. Soon after, for large city population,  once enough people have been infected, the future evolution of the pandemic is closely approximated by its mean field limit with a random starting state. We build upon these insights to develop a shifted, scaled and restart based algorithm that accurately evaluates the ABS's performance  using  a much smaller model while carefully reducing the bias that may otherwise arise. We theoretically support 
the proposed algorithm through an asymptotic analysis where the population size increases to infinity.
\end{abstract}


\section{Introduction} 
  \label{Introduction}
COVID-19 is an ongoing pandemic caused by the SARS-COV 2 virus that began in December 2019 in Wuhan and through more infectious and dangerous variants continues to rage into its third year.  The reported cases worldwide are around 496 million and reported fatalities 6.17 million as of April 08, 2022. 
Early on, countries across the world
implemented a variety of non-pharmaceutical interventions such as testing, tracing, tracking, isolation, quarantining, enforcement of mask wearing, social distancing,
shutting public transportation, lockdowns, etc. to reduce the disease spread.
Later, since December 2020, vaccines became a key weapon in fight against the virus.
Predicting the consequences of presence of different  
variants and the vaccines, as well as of non medical interventions is important
  in selecting the preferred course of action.  Epidemiologists often resort to sophisticated mathematical
 models that predict the related statistics of health indicators including
 the total number of infected,
 hospitalisations, critical cases
  and fatalities as a function of time. 
  
  The classical `aggregate'  models for disease spread 
  capture the mean field dynamics of the disease within the population
  through simple ordinary differential equations (ODEs). 
  These models  typically involve a few parameters that need to be estimated from data, 
 are easily solved using a numerical ODE solver 
 and are good at highlighting macroscopic trends
 (see, \cite{kermack1927contribution} for seminal work.
 A few recent references amongst thousands are \cite{SIR1}, \cite{SIR3}, \cite{SIR4}, \cite{SIR5}).
 However, such aggregate models often 
 are inadequate in accurately capturing the initial uncertainty inherent in the  disease spread and its impact on future disease evolution, 
 impact of more detailed, localised
 and complex interventions,
 emergence of new variants, and their co-evolution
 with the existing ones,  and other important features such as detailed implementation of a vaccination strategy. 
  
  \bigskip
    
{\noindent \bf Agent-based simulators and their advantages:} In this paper we focus on another popular class of models, namely 
the agent-based simulators (ABS) (See \cite{hunter2017taxonomy}, \cite{ferguson2020report}). 
In such models, a  synthetic copy of the region is constructed on a computer
that attempts to  capture the population interaction spaces and detailed disease spread as well as disease spreading interactions
as they evolve in time.
Typically, each individual in the region is modelled as
an agent, so that total number of agents equals
the total region population. 
The constructed individuals reside in homes, children may go 
to schools, adults may go to work. Individuals also engage with each other in
community spaces (to capture interactions in marketplaces, restaurants,
public transport,
and other public places).
Homes, workplaces, schools sizes and locations and  individuals associated with them,  
their gender and age,  are created to match the region census 
data and are distributed to match its geography. Government policies, such as partial, location specific  lock downs for small periods of time,
case isolation of the infected and home  quarantine of their close contacts,
closure of schools and colleges, partial openings of workplaces, etc. are easily modelled. Similarly, variable compliance behaviour in different segments of 
the population that further changes with time, is easily captured
in an ABS.  Further, it is easy 
to introduce new variants as they emerge,  the individual vaccination status, as well as the protection offered by the vaccines
against different variants as a function of
the evolving state of the epidemic and individual characteristics, such as age, density of individual's interactions, etc (see \cite{May_report_2021}). 
Thus, this microscopic level  modelling flexibility allows ABS to become an effective strategic and operational tool
to manage and control the  the disease spread.  
 See \cite{ferguson2020report} for an ABS used 
for UK and USA related studies specific to COVID-19,
\cite{gardner2020intervention} for a COVID-19 study on Sweden, \cite{City_Simulator_IISc_TIFR_2020}, \cite{October_report_2020}
for a study on Bangalore and Mumbai in India.
See, e.g.,   \cite{halloran2008modeling}  and \cite{hunter2017taxonomy} for an
overview of different agent-based models.

\bigskip 

{\noindent \bf Key drawback:} However, when a reasonably large population is simulated, especially over a long time horizon, an ABS can take huge computational time 
and this is its key drawback.  This becomes particularly
prohibitive when multiple runs are needed using different parameters. 
For instance, any predictive analysis involves simulating a large number of scenarios
to provide a comprehensive view of potential future sample paths. 
In model calibration, 
the key transmission parameters  are fit to
the infection data during the early stages of the disease, requiring many 
computationally demanding simulations. Many other important 
parameters such as infectiousness of new variants
are similarly selected to match the observed data over relevant time horizons, again requiring massive computational
effort. 


\bigskip


{\noindent \bf Key contributions:} The key contribution of this paper is to develop a {\bf shift-scale-restart} algorithm (described later)  that  carefully exploits the closeness
of the underlying infection process (process of number infected 
of each type at each time), first to a multi-type super critical branching process, and then, suitably normalised, 
 to the mean field limit of the infection process,
so that the output from the smaller model accurately 
matches the output from the larger one. 
Essentially, both the smaller and the larger model, with identical initial conditions of small number of infections
evolve similarly in the early days of the infection growth.  
Interestingly, in a super-critical branching process, while the number of infections grows exponentially, the proportions of the different types infected quickly stabilises, and this allows us to shift a scaled
path from a smaller model to a later time with negligible change in the underlying distribution.
Therefore, once there are enough infections in the system, output from the smaller model, when scaled, matches that from the larger model at a later `shifted' time.
This shifting and scaling of the paths from the smaller model does good job of representing output from the larger model when there are no interventions
to the system. However, realistically, government intervenes and population mobility behaviour changes with increasing infections. 
To get the timings of these interventions right, we restart the smaller model and synchronise the timings of interventions in the shifted and scaled path to the actual timings in the original path of the larger model.

We present numerical results where our strategy is implemented on a  model
for Mumbai with 12.8 million population that realistically captures  interactions at home, school, workplace and community as well as mobility
restrictions through  interventions such as lockdown, home quarantine, case isolation, schools closed, limited attendance at workplaces, etc.
Using our approach, we more or less exactly replicate 12.8 million population model for Mumbai with all its complexities, using only 1 million people model, providing an almost 12.8 times speed-up.

Further, we  provide theoretical support for the proposed approach through an asymptotic analysis where the population size $N$ increases to infinity. 
We
show that early on till time $\log (N/\log N)/\log (\rho)$, where $\rho$ is the exponential epidemic growth rate in early stages,   the epidemic process is well approximated by an associated  branching process.
Then, after time  $\log ( \epsilon N)/\log (\rho)$ for any $\epsilon >0$, the epidemic process is closely approximated by its mean field
limit that can be seen to follow a large dimensional discretized ODE.  While, our simulation model allows each person's state to lie in an uncountable state space,  our theoretical analysis is conducted under somewhat simpler finite state assumptions, that subsumes  general  compartmental models. 
As is empirically evident, the results hold more generally. However, the analysis required is more intricate and requires further research. 
We also identify the behaviour of simple compartmental models such as susceptible-infected and recovered (SIR) and some generalisations 
in the early branching process phase. 
 
 \textbf{The remaining paper is structured as follows:} 
 To illustrate the key ideas practically, we first show the implementation of our algorithm  on a realistic model of Mumbai in Section~\ref{section:big_picture}.
Then in Section \ref{ABS_brief}, we briefly   summarise our agent based simulator.  In Section~\ref{ssr_algo} we 
spell out the shift-scale-restart algorithm. 
We provide theoretical asymptotic analysis supporting the efficacy of the proposed algorithm in Section \ref{Theoretical_results}.
The technical proofs are presented in the  Section \ref{appendix}, where for completeness 
we also spell out the input data used in our simulation model for Mumbai. \footnote{A three page extended abstract of this paper appeared in \cite{sigmetricsposter}.}
 
 %As we discuss later, when features that depend upon geographical distances are involved, such as  geographically fixed neighbourhood cells used to model containment zones, care is needed to ensure that susceptible individuals observe the same infection rate both in smaller as well as larger models.


\section{Speeding up ABS: The big picture}
\label{section:big_picture}

A naive approach to speed up the ABS maybe to use a representative smaller population model and scale up the results. Thus, for instance, while a realistic model for Mumbai city may have 12.8 million agents (see \cite{May_report_2021}), we may construct a sparser Mumbai city having, say, a million agents,  that matches the bigger model in essential features, so that, roughly speaking, in the two models each infectious person contributes the same total infection  rate to all susceptibles at each time. The output numbers from the smaller model may be scaled by a factor of 12.8 to estimate the output from the larger model. We observe, somewhat remarkably, that this naive approach is actually accurate  if the initial seed infections in the smaller model (and hence also the larger model) are large, say, of the order  of thousands, 
and are identically distributed in both the smaller and the larger model 
(see Figure \ref{12800_infections} where the number exposed are plotted under a counter-factual no interventions scenario. The comparative statements hold
equally well for other statistics such as the number infected, hospitalised, in ICUs and deceased). 
The rationale is that in this setting both the smaller and the larger model have sufficient infections so that 
the proportion of the infected population in both the models
well-approximate their identical mean field limits.
 
 
 \begin{figure}
      \centering
 \includegraphics[width=\linewidth]{Graphs/12800_infections.png}   
    \caption{Scaled no. exposed in the smaller model match the larger model when we start with large, 12800, no. of infections. } 
    \label{12800_infections}
  \end{figure}
 
However, modelling initial randomness in the disease spread is important for reasons including ascertaining the distribution of when and where an outbreak may be initiated, the probability  that some of the initial infection clusters die-down, getting an accurate  distribution of geographical spread of infection over time, 
capturing the  intensity of any sample path (the random variable $W$ in the associated branching process, described in Theorem~\ref{theorem_extending_branching_process}), etc.
These are typically captured by setting the initial infections to a small number, say, around a hundred, and the model is initiated at a well chosen time (see \cite{City_Simulator_IISc_TIFR_2020}). 
In such settings,  we observe that the scaled output from the smaller model 
(with proportionately lesser initial infections) is noisy and biased so that the simple scaling fix no longer works (see Figure \ref{128_infections}. Later 
in Section ~\ref{heuristic_initial}, we explain in a simple setting why the scaled smaller model is biased and reports lower number of infections compared to the larger model in the 
early infection spread phase).
 In fact, we observe that in the early days of the infection, the smaller and the larger model with the \textbf{same number of initial infections}, 
 similarly clustered, behave more or less identically (see Figure \ref{100_infections_upto_day_35}), so that the smaller model  with the unscaled number of initial seed infections 
provides an accurate approximation to the larger one. Here again in the early phase in the two models, each infectious person contributes roughly the same total infection  rate to susceptibles 
at home, workplace and community.  
Probabilistically this is true because early on, both the models closely approximate an identical multi-type branching process. Shift-scale-restart algorithm outlined in Section \ref{ssr_algo} exploits these observations to speed up the simulator. We briefly describe it below.

\begin{figure}
      \centering
      
 \includegraphics[width=\linewidth]{Graphs/128_infections.png}   
    \caption{Scaled no. exposed in smaller model do not match the larger model when we start with few, 128, no. of infections. }
    \label{128_infections}
  \end{figure}

{\noindent \bf Fixing ideas:} Suppose that for Mumbai with an  estimated population of $12.8$ million, a $12.8$ million agent model is seeded 
with $100$ randomly distributed infections on day zero. To get the statistics of interest such as expected hospitalisations and fatalities over time, instead of running 
the  $12.8$ million agent model, we start a $1$ million agent model seeded
with $100$ similarly randomly distributed infections at day zero and  generate a complete path for the requisite duration.
To get the statistics for the larger model, we first observe that under the no-intervention scenario,  the output of the smaller model more or less exactly matches that of the larger model
for around first $35$ days (Figure \ref{shift_scale_100_infections}). 
As our analysis in Section \ref{Theoretical_results} suggests, the two models closely approximate the associated branching process till
time $\frac{\log(N/(i\log N))}{\log \rho}$ where $N$ denotes the population of the smaller model and equals 1 million, $i$ denotes the number exposed at time zero and equals 100,
and $\rho$ denotes the exponential growth rate in the early fatalities and is estimated from fatality data to equal 1.21. The quantity  $\frac{\log(N/i \log N)}{\log \rho}$ is estimated to equal 
34.5 
 suggesting that both are close to the underlying branching process  around day  35.  After this initial period of around $35$ days, the city has an average of 
$x$ thousand infections. We then determine  the day when the city had $\frac{x}{12.8}$ thousand infections.
This turns out to be  day $21.5$ in our example. We take the path from day 21.5 onwards, scale it by a factor of 12.8, and concatenate it to the original path
starting at day $35$.  Theoretical justification for this time shift comes from the branching process theory,
where while a super-critical multi-type branching process can be seen to grow exponentially with a sample path dependent intensity, the relative proportions  amongst types along each sample path stabilise fairly quickly and become more-or-less stationary (see Theorem \ref{stbp_theorem}). 
This shifted and scaled output after 35 days matches that of the larger model remarkably well. See Figure    \ref{shift_scale_100_infections} where the generated infections from the larger $12.8$ million model
and the shifted and scaled smaller $1$ million model are compared.  The choice of day 35 is not critical above. Similar results would be achieved
if we used lesser, as low as  25 days in the original model. 

In a realistic setting, administration may intervene once the reported cases begin to grow. Suppose in the above example, an
intervention happens on day 40.  (This is not unreasonable, as in modelling Mumbai, 
our calibration exercise had set the day zero to February 13, 2020 (see, \cite{October_report_2020}), the resulting infections and  reported cases reached worrying levels 
around the second week of March. Restrictions in the city were  imposed around March 20, 2020.)
In that case, the shifted and scaled path from the day 21.5 would need to have the restrictions imposed on day 26.5 (so that it approximates day 40 
for the larger model). We achieve this by using the first generated a path till day 35, computing the appropriate scaled path time (21.5 days, in this case),
 and then, using common random numbers, restarting an identical path from time zero that has restrictions imposed from day 26.5. This path is scaled from day 21.5 onwards and concatenated to the original path at day 35. 
 Later in Section \ref{branching_results} we note that the restarted path need not use common random numbers from the original path. Even if it is generated independently, we get very similar output.
 
To summarise, our shift-scale approach works because, early on when the small model is close to the branching process, shifting across time is valid
as the proportions across types do not distort significantly. Thereafter, the scaling property holds (smaller model, suitably scaled, well approximates the larger model)
even when the number infected increase to become of the same order as the susceptible, since both the smaller and the larger model closely approximate 
their common mean field limit evolution process (see Theorem \ref{deterministic_theorem} for a theoretical justification). 
 
 
 %\textbf{Figure (Daksh) illustrates this when the intervention on day 40 corresponds to home quarantine.  We later discuss how the smaller model is adjusted to match the larger one with more complex interventions such as containment zones, and show that the fully calibrated larger model output closely matches that from the  adapted smaller model.}
 
 


 % \begin{figure}
 %   \centering
 %   \includegraphics[width=\linewidth]{Graphs/128000_infections.png%}
 % \caption{ Initial Infections 128000 and 10000 respectively in %12.8 million city and 1 million city. .}
%\label{128000_infections}
 % \end{figure}
 
 
 

  
  
  \begin{figure}
      \centering
      
 \includegraphics[width=\linewidth]{Graphs/100_infections_upto_day_35.png}   
    \caption{Smaller and larger model are essentially identical initially when we start with same number of few, 100, no. of infections }
    \label{100_infections_upto_day_35}
  \end{figure}
 


\begin{figure}
    \centering
    \includegraphics[width=\linewidth]{Graphs/shift_scale_100_infections.png}
  \caption{Shift and scale smaller model (no. of exposed) matches the larger model under no intervention scenario. }
\label{shift_scale_100_infections}
  \end{figure}
  
  
  \begin{figure}
    \centering
    \includegraphics[width=\linewidth]{Graphs/hosp_no_int.png}
  \caption{Shift and scale smaller model (no. of hospitalised) matches the larger model under no intervention scenario. }
\label{hosp_no_int}
  \end{figure}
  
  
  \begin{figure}
    \centering
    \includegraphics[width=\linewidth]{Graphs/fatal_no_int.png}
  \caption{Shift and scale smaller model (no. of cumulative fatalities) matches the larger model under no intervention scenario. }
\label{fatal_no_int}
  \end{figure}

Figures \ref{shift_scale_100_infections}, \ref{hosp_no_int} and \ref{fatal_no_int} compare the number exposed, the number hospitalised and the number of cumulative fatalities in a 12.8 million Mumbai city simulation (in no intervention scenario) with the estimates from the shift-scale-restart algorithm applied to the smaller 1 million city. Figures \ref{12800_infections} to \ref{fatal_no_int} are  under no intervention scenario for Mumbai
as described in Section \ref{Numerical Parameters Section}. Figures \ref{shift_scale_restart_1}, \ref{hosp_hq_40} and \ref{fatal_hq_40} compare
the  number exposed, the number  hospitalised, and the number of cumulative fatalities in a 12.8 million Mumbai city simulation (intervention: home quarantine from day 40) with estimates from the shift-scale-restart algorithm applied to the smaller 1 million city.
  Figure \ref{shift_scale_restart_real_intervention} compares the exposed population process for 
  the 12.8 million population Mumbai model with the smaller 1 million  one  as per our algorithm  
under realistic interventions (lockdown, case isolation, home quarantine, masking etc.) introduced at realistic times, as implemented in \cite{report2} using similar parameters, for 250 days. We see that the smaller model faithfully replicates the larger one with negligible error.
%Figure  \ref{ssr_real_int_with_new_strain} similarly compares the two models
%for a longer 470 days horizon. This includes the introduction of delta variant (see, \cite{WSC} for methodological details for introducing the delta variant) as well as the pre-delta variant phase 
%where the number of infections in Mumbai was ebbing.
%exposed population process for Mumbai with the smaller 1 million  city as per our algorithm  
%under realistic interventions (lockdown, case isolation, home quarantine, masking etc.) and a new infectious strain (delta strain) introduced at realistic times. As noted earlier, the input data used in the experiments is spelled out in Section \ref{Numerical Parameters Section}.


\begin{figure}
    \centering
    \includegraphics[width=\linewidth]{Graphs/Shift_scale_restart_hq_40_days.png}
  \caption{Shift-scale-restart smaller model (no. of exposed) matches the larger model (intervention: home quarantine from day 40). }
\label{shift_scale_restart_1}
  \end{figure}
  
  
  \begin{figure}
    \centering
    \includegraphics[width=\linewidth]{Graphs/hosp_hq_40.png}
  \caption{Shift-scale-restart smaller model (no. of hospitalised) matches the larger model (intervention: home quarantine from day 40). }
\label{hosp_hq_40}
  \end{figure}
  
  
  \begin{figure}
    \centering
    \includegraphics[width=\linewidth]{Graphs/fatal_hq_40.png}
  \caption{Shift-scale-restart smaller model (no. of cumulative fatalities) matches the larger model (intervention: home quarantine from day 40). }
\label{fatal_hq_40}
  \end{figure}


\begin{figure}
      \centering
      
 \includegraphics[width=\linewidth]{Graphs/shift_scale_restart_real_intervention.JPG}   
   \caption{Shift-scale-restart smaller model match the larger one under real world interventions over 250 days. }
    \label{shift_scale_restart_real_intervention}
  \end{figure}

 %  \begin{figure}
  %    \centering
      
 %\includegraphics[width=\linewidth]{Graphs/ssr_real_int_with_strain.png}   
 %  \caption{Shift-scale-restart smaller model match the larger one under real world interventions over 470 days with a new delta strain after 350 days. }
  %  \label{ssr_real_int_with_new_strain}
 % \end{figure}

\section{Agent Based Simulator} 

\label{ABS_brief}
In this section, we informally describe the main drivers of  the dynamics in our infection spread model. A more detailed discussion can be seen in \cite{City_Simulator_IISc_TIFR_2020}. 

The model consists of individuals and various interaction spaces such as households, schools, workplaces and community spaces.
 Infected individuals interact with susceptible individuals in these interaction spaces. The number of individuals living in a household, their age, whether they go to school or work or neither, schools and workplaces  size and  composition all have distributions that may be set to match the available data. The model proceeds in discrete time steps of constant width $\Delta t$ (six hours in our set-up). At a well chosen time zero, a small number of individuals can be set to either exposed, asymptomatic, or symptomatic states, to seed the infection. At each time $t$, an infection rate $\lambda_n(t)$ is computed for each susceptible individual $n$ based on its interactions with other infected individuals in different interaction spaces (households, schools, workplaces and  community).  In the next $\Delta t$ time, each susceptible individual moves to the exposed state with probability $1 - \exp\{ - \lambda_n(t) \cdot \Delta t\}$, independently of all other events. Further, disease may progress independently in the interval $\Delta t$ for the population already afflicted by the virus. The computation of $\lambda_n(t)$ and the probabilistic dynamics of disease progression are briefly summarized below under simplified assumptions (details available in \cite{City_Simulator_IISc_TIFR_2020}, also see \cite{ferguson2020report}). Simulation time is then incremented to $t + \Delta t$, and the state of each individual is updated to  reflect the new exposures, changes to infectiousness, hospitalisations, recoveries, quarantines, etc., during the period $t$ to $t + \Delta t$.  The overall process repeats incrementally until the end of the simulation time. 


\noindent \textbf{Computing $\lambda_n(t)$ :} A susceptible individual $n$ at any time $t$ receives a total infection rate $\lambda_n(t)$ which is sum of the infection rates $\lambda_n^{h}(t)$ (from home), $\lambda_n^{s}(t)$ (from school), $\lambda_n^{w}(t)$ 
(from workplace) and  $\lambda_n^{c}(t)$  (from community)
 coming in from infected individuals in respective interaction spaces of individual $n$. 
\[\lambda_n(t)= \lambda_n^{h}(t)+\lambda_n^{s}(t)+\lambda_n^{w}(t)+\lambda_n^{c}(t).\]

Briefly, the transmission rate ($\beta$) of virus by an infected individual in each interaction space is the expected number of eventful (infection spreading) contact opportunities with all the individuals in that interaction space. It accounts for the combined effect of frequency of meetings and the probability of infection spread during each meeting. 
 An infected individual can transmit the virus in the infective (pre-symptomatic or asymptomatic stage) or in the symptomatic stage. Each individual has two other parameters: a severity variable (individual attendance in school and workplace depends on the severity of disease) and a relative infectiousness variable, virus transmission is related linearly to this. Both bring in heterogeneity to the model. 
 
 Specifically, let $e_{n'}(t) = 1$ when an individual $n'$ can transmit the virus at time $t$, $e_{n'}(t) = 0$ otherwise. To keep notation simple, we avoid describing the details of individual infectiousness, severity, age dependent mobility and community density factor (see \cite{City_Simulator_IISc_TIFR_2020} for details). Let $\beta_h$, $\beta_s$, $\beta_w$, and $\beta_c$  denote the transmission coefficients at home, school, workplace and  community spaces, respectively. A susceptible individual $n$ who belongs to home $h(n)$, school $s(n)$, workplace $w(n)$ and community space $c(n)$ sees the following infection rates at time $t$ in different interaction spaces:
 

     $\lambda_n^{h}(t)$ is the average transmission rate coming in to individual $n$ from each infected individual in his home. 
     \[\lambda_n^{h}(t)=\beta_{h}   \frac{\sum_{n' : h(n') = h(n)} e_{n'}(t)}{n_{h(n)}}.   \]
    
     $\lambda_n^{s}(t)$ and $\lambda_n^{w}(t)$  are also computed similarly based on infected individuals in respective interaction spaces of the individual $n$. 
%     \[\lambda_n^{school}(t)=\sum_{n' : s(n') = s(n)} \frac{1}{n_{s(n)}} \cdot I_{n'}(t) \beta_{s} \]
 %    \[\lambda_n^{workplace}(t) = \sum_{n' : w(n') = w(n)} \frac{1}{n_{w(n)}} \cdot I_{n'}(t) \beta_{w}\]
    
     %Different wards in the city constitute different communities. Community infection rate ($ h_{c'}(t)$) of each community  is sum of transmission rate from all the infected individuals of the community weighted inversely based on the distance between the individual and community centre. $\lambda_n^{community}(t)$ seen by an individual $n$ in community $c$ is sum of the community infection rates from different communities of the city weighted inversely based on the distance between the two communities and distance between individual $n$ and its community center. An age dependent factor determining the mobility of the individuals in the community is also accounted in $\lambda_n^{community}(t)$. 

  
  Different wards in the city constitute different communities. For Mumbai, since slums form more than half the population
 and are extremely dense, we further divide each ward into slum and non-slum community.
 To compute infection rate to an individual from the community, we first compute the infection rate $h_{c'}(t)$ for each community $c'$ at a given time $t$.
 This is set to the 
 sum of transmission rate from all the infected individuals of the community assigned a weight that is
 proportional to the strictly decreasing function of the distance between the individual and the community centre. Specifically,
   
  \begin{equation} \label{eqn:comm1}
   h_{c'}(t) = \beta_{c} \left( \frac{\sum_{n': c(n') = c'} f(d_{n',c'}) \cdot e_{n'}(t) }{\sum_{n'} f(d_{n',c'})}  \right)
   \end{equation}
where  $f(d)$ is a distance kernel that is strictly decreasing in $d$. As in  \cite{ferguson2005strategies}, one may select
 $f(d) = 1/(1+(d/a)^b)$ and $d \ll a$ for appropriately chosen non-negative constants $a$ and $b$.
 Note that $h_{c'}(t)$ is dominated by $\beta_{c}$ and it equals $\beta_{c}$ only when everyone
 in the community $c'$ is infected.
 
 
The infection rate  $\lambda_n^{c}(t)$ seen by an individual $n$ in community $c$ is set to sum of the community infection rates from different communities of the city multiplied with weights that are again proportional to  the strictly decreasing function of the distance between  the two communities. 
This quantity is further adjusted for the  distance between individual $n$ and its community centre. Specifically,
 \begin{equation} \label{eqn:comm2}
 \lambda_n^{c}(t)= \frac{ f(d_{n,c(n)})}
  {\sum_{c'}f(d_{c(n),c'})} \sum_{c'} f(d_{c(n),c'})
  h_{c'}(t).
  \end{equation}
  
  Again observe that, if  $f(d_{n,c(n)})<1$, then $\lambda_n^{c}(t) < \beta_c$.  Further, an age dependent factor determining the mobility of the individuals in the community may  also be accounted in $\lambda_n^{c}(t)$
(see \cite{City_Simulator_IISc_TIFR_2020}). 
% In a city such as Mumbai, local trains can be thought of as another community where people travelling for work come together. We omit the discussion 
% on modelling trains here although they are  captured in our experiments.
 Containment zones were an important government intervention where as the cases in a region rise, they are  contained through localised
 movement restrictions. We omit the associated modelling details (see \cite{harsha2020covidmumbai}), although they are captured in our experiments. In \cite{City_Simulator_IISc_TIFR_2020} an implementation of contact tracing methodology is discussed. However, this has minimal effect on disease spread, and is not considered in the code we use for experiments in this paper.
 



%\begin{eqnarray}
%\lambda_n(t)
%  & = & \sum_{n' : h(n') = h(n)} \frac{1}{n_{h(n)}} \cdot I_{n'}(t) \beta_{h} \nonumber ~
%   + \sum_{n' : s(n') = s(n)} \frac{1}{n_{s(n)}} \cdot I_{n'}(t) \beta_{s} 
%   \nonumber \\
%  & & +~ \sum_{n' : w(n') = w(n)} \frac{1}{n_{w(n)}} \cdot I_{n'}(t) \beta_{w}  \nonumber ~
%   +~ \frac{\cdot f(d_{n,c(n)})}
%  {\sum_{c'}f(d_{c(n),c'})} \sum_{c'} f(d_{c(n),c'})
%  h_{c'}(t) \label{eqn:rate}
%\end{eqnarray}
%where
%\begin{equation}
% \label{eqn:community-interaction}
% h_{c'}(t) = \left( \frac{\sum_{n': c(n') = c'} f(d_{n',c'}) \cdot I_{n'}(t) \beta_{c}}{\sum_{n'} f(d_{n',c'})}  \right)
%\end{equation}

%The expression \eqref{eqn:rate} can be viewed as the rate at which the susceptible individual $n$ contracts the infection at time $t$. Each of the components on the right-hand side indicates the rate from home, school, workplace, and community. The additional quantities, over and above what we have already described, are as follows. 
%The quantity $h_{c'}(t)$ represents the total community infection rate from infected individuals in ward $c'$. As described above, each individual contributes in a distance-weighted way in how an individual in a ward $c'$ affects another individual in another ward $c$.


%\textbf{We currently consider only the larger network interaction spaces such as homes, workplaces, schools and community. However, numerical results also hold when we consider smaller networks such as neighbourhood, class, project and random community, respectively with some modifications}

\bigskip

\noindent \textbf{Disease progression :} Our model of COVID-19 progression is based on descriptions in \cite{verity2020estimates} and \cite{ferguson2020report}. An individual may have one of the following states: susceptible, exposed, infective (pre-symptomatic or asymptomatic), recovered, symptomatic, hospitalised, critical, or deceased. Briefly, an individual after getting exposed to the virus at some time observes an incubation period which is random with a Gamma distribution. Individuals are infectious for an exponentially distributed period. This covers both presymptomatic transmission and possible asymptomatic transmission. We assume that a third of the patients recover, these are the asymptomatic patients; the remaining develop symptoms. Individuals either recover or move to the hospital after a random duration that is exponentially distributed (see Section \ref{Numerical Parameters Section} for model input data details). The probability that an individual recovers depends on the individual's age. While hospitalised individuals may continue to be infectious, they are assumed to be sufficiently isolated, and hence do not further contribute to the spread of the infection. Further progression of hospitalised individuals to critical care and further to fatality is also age dependent. 

\bigskip

\noindent \textbf{Public health safety measures (PHSMs):} We introduce methodologies to model different PHSMs (or Interventions) such as lockdowns, home quarantine, case isolation, social distancing of elderly population, mobility restrictions, masks etc. See \cite{City_Simulator_IISc_TIFR_2020} for details on how these interventions are modelled in the simulator. Table \ref{interventions} in Section \ref{Numerical Parameters Section} summarizes some of the key interventions implemented. The PHSM's mentioned above when implemented put some restrictions on the individual's mobility. However, it is often the case that when several restrictions are in place, only a fraction of the population comply with these restrictions. Therefore, we restrict the movement of only the individuals in the compliant fraction of households in the city. 



 
%This factor matches the high prevalence of 56\% observed in Mumbai slums and lower prevalence of 16\% in non slums in July 2020 \shortcite{Mumbai_sero2020}. The contact rates are calibrated to match the observed growth of fatalities in the city till April 10, 2020, and to have roughly equal contribution of infections from the household, community and workplace networks (including the subnetworks) in the ``no-intervention'' scenario as mentioned in ~\shortcite{City_Simulator_IISc_TIFR_2020}. Compliance rates are again selected to match the observed fatality data early on (in April-June 2020) and later it is varied based on Google mobility report. We refer the reader to \shortcite{October_report_2020} for the parameter values. 
%(see Figure \ref{fig:beta-values})
 
\section{Shift-scale-restart algorithm} 
\label{ssr_algo}


Let $\mu_0(N)$ denote the initial distribution of the infected population at time zero in our simulation model with population $N$
and let the  simulation run for a total of $T$ time units.  E.g., for Mumbai at a suitably chosen time 0, we select 100 people at random from the
non-slum population and mark them as exposed. (Since initially the infection came from international travellers flying into Mumbai, it is reasonable to assume that most of them were residing in non-slums).   
The simulation dynamics may be summarised in Algorithm 1. 
\begin{algorithm}
\caption{Simulation Dynamics}\label{alg:base}
\begin{algorithmic}[1]
 \State  At $t=0$, start the simulation with $I_0$ infections distributed as per $\mu_0(N)$. 
 
 \While {$t < T$}
 \State For each susceptible individual
 $n$, calculate $\lambda_n(t)$.  Its status then changes to exposed 
 with probability $1-\exp({-\lambda_n(t)})$.
 \State All individuals in some state other than susceptible, independently transition to another state as per the disease progression 
 dynamics. 
 \State $t \gets t+1$.
 \EndWhile
 \State The above simulation is independently repeated many times and average of performance measures such as number exposed, number infected, number hospitalised, and number of fatalities as a function of time are reported.
\end{algorithmic}
\end{algorithm}



For $k, N \in \mathbb{N}$, $k>1$, let $kN$ be the number of individuals in the larger city, and $N$ in the  smaller city. 
Roughly speaking, the larger city has $k$ times more homes, schools and workplaces compared to the smaller city.
The joint distribution of people in  homes, schools or workplaces is unchanged, and 
transmission rates $\beta_h$, $\beta_s$, $\beta_w$ and $\beta_c$ are unchanged.

When we initiate both the larger as well as the smaller city with a same few and well spread infections, the disease spread similarly
in homes, schools and workplaces. To understand the disease spread through communities, 
for simplicity assume that there is a single community and $f$ equals 1 in (\ref{eqn:comm1}) and (\ref{eqn:comm2}).
It is easy to see that each susceptible person sees approximately $1/k$ times the community infection rate in the larger city
compared to the smaller city. On the other hand, the larger city has $k$ times more susceptible population.
This is true even when there are more wards and for a general distance function $f$.
Therefore, early on in the simulation, the total number of people getting infected through communities is also 
essentially identical between the larger and the smaller city, and the infection process in the two cities evolves
very similarly.

 Let 
$t_{S}$ denote  the time till the two cities evolve essentially identically (as seen empirically and suggested by theoretical analysis, 
this is close to  $\log_\rho N^*$ for  $N^* = N/\log{N}$. Here  $\log_\rho m = \log {m}/\log \rho$ for any $m\in \Re^+$ and $\rho$ denotes the initial infection exponential growth rate.)

First consider the contrafactual no intervention scenario where the population intermingling behaviour does not change
even as the disease spreads through the population. In this setting we propose that a shift and scale Algorithm~\ref{alg:cap},
that builds upon Algorithm \ref{alg:base}, be applied to the smaller city to generate output that resembles the larger city.
 Algorithm~\ref{alg:cap} is graphically illustrated in Figure \ref{SSR_1_final_1}.

\begin{algorithm}
\caption{Shift and scale algorithm in no-intervention setting}\label{alg:cap}
\begin{algorithmic}[1]
 \State   At $t=0$, start the simulation with $I_0$ infections distributed as per $\mu_0(N)$.
 Generate the simulation sample path 
 $[y_1,y_2,...,y_{T}]$ 
 where $y_t$ denotes the statistics of the affected population (e.g., number exposed, number hospitalised, 
 number of fatalities)  at time $t$. 
  \bigskip
 \State Suppose there are $x$ infections at $t_{S}$, determine an earlier time $t_{x/k}$ in the
  simulation when there where approximately $\frac{x}{k}$ infections in the city. 
 \bigskip
 \State  The statistics of affected for the larger city is then obtained as  
 \[
 [y_1,y_2,...,y_{t_S}, k \times y_{t_{x/k}+1},...,k \times y_{T-( t_S-t_{x/k})}].
 \]
 %Scale the path of the second (restarted) simulation from time $t_{\frac{x}{k}}$ onwards by a factor of $k$, and append this scaled path to the first simulation of the smaller city after $t_{min}$. Therefore the whole path is
 %\Comment{To get the large city path from these two simulations of the smaller city}
 \bigskip
\State 
The above simulation is independently repeated many times and average of performance measures such as number exposed, number infected, number hospitalised, and number of fatalities as a function of time are reported.
\end{algorithmic}
\end{algorithm}

In a realistic scenario, as the infection spreads,  the administration will intervene and impose mobility restrictions.
Thus, our simulation adjustments to the small city need to account for the timings of these interventions accurately. 
Let $t_{I}$ denote the first intervention time (e.g., lockdown; typically
after $\beta \log_\rho N$ time for small $\beta \in (0,1)$).  Let $t_{min} \approx  \min\{t_{I},t_{S}\}$. 
We need to restart our simulation to ensure that the shift and scaled path
incorporates the intervention at the correct time. The Algorithm~\ref{alg:cap2}
achieves this and is graphically illustrated in Figure  \ref{SSR_2_final_1}. 

\begin{algorithm}
\caption{Shift, scale and restart algorithm}\label{alg:cap2}
\begin{algorithmic}[1]
\State   At $t=0$, start the simulation with $I_0$ infections distributed as per $\mu_0(N)$.
 Generate the simulation sample path  
 $[y_1,y_2,...,y_{t_{min}}]$ where $y_t$ denotes the statistics of the affected population (e.g., number exposed, number hospitalised, 
 number of fatalities)  at time $t$. 
 \bigskip
 \State Suppose there are $x$ infections at $t_{min}$, determine an earlier time $t_{x/k}$ in the
  simulation when there where approximately $\frac{x}{k}$ infections in the city. 
 \bigskip
 \State Restart 
 a new simulation of the city using common random numbers, but with the intervention introduced 
 at time $t_{\frac{x}{k}}+t_{I}-t_{min}$. Simulate it upto time $T-(t_{min}-t_{x/k})$. 
Denote the time series of statistics of the affected population in the restart simulation
by \[z_1,z_2,...,z_{t_{x/k}},...,z_{T-(t_{min}-t_{x/k})}.\]
 
 \bigskip
 \State  The approximate statistics of the affected population for the larger city is then obtained as   
 \[
 [y_1,y_2,...,y_{t_{min}},k \times z_{t_{x/k}+1},...,k \times z_{T+t_{x/k}-t_{min}}]
 .\]
  \bigskip

\State 
The above simulation is independently repeated many times and average of performance measures such as number exposed, number infected, number hospitalised, and number of fatalities as a function of time are reported.
 %Scale the path of the second (restarted) simulation from time $t_{\frac{x}{k}}$ onwards by a factor of $k$, and append this scaled path to the first simulation of the smaller city after $t_{min}$. Therefore the whole path is
 %\Comment{To get the large city path from these two simulations of the smaller city}
\end{algorithmic}
\end{algorithm}


As we see empirically, 
and as is suggested by Proposition \ref{initial_proposition}, in Algorithms \ref{alg:cap} and \ref{alg:cap2}, evolution after time $t_{x/k}$, the infection process is 
 more or less deterministic.  
 
 
 %Below we heuristically argue in a simple SIR (susceptible-Infected-Recovered) setting
 %why the output from our algorithm in the smaller city matches the output from the larger city in this mean field phase.  
 
 
 \begin{figure}
      \centering
  \includegraphics[width=\linewidth]{Graphs/SSR_1_final_2.PNG}   
    \caption{ Shift and scale under no intervention scenario} 
    \label{SSR_1_final_1}
  \end{figure}
  
  \begin{figure}
      \centering
  \includegraphics[width=\linewidth]{Graphs/SSR_2_final_2.PNG}
    \caption{ Shift, scale and restart algorithm} 
    \label{SSR_2_final_1}
  \end{figure}
  
 

 

 
 
 





  

 
 
\section{Asymptotic Analysis of Epidemic Process} 
\label{Theoretical_results}

In the SSR algorithm, theoretical justification is needed for the fact that early on in the small city simulation, we could take a path at one time period, scale it, and stitch it to the path at another appropriately chosen time period to accurately 
generate a path for the larger city.   We provide this through analyzing our city in an asymptotic regime as the city population
$N$ increases to infinity. To bring out the key observations simply, we consider a simpler model where the interaction spaces of homes, workplaces and schools are ignored and only a single community interaction space is retained.

 In the following analysis, we first review standard multi-type branching process in Section \ref{define_standard_MTBP}. We then define an epidemic process in Section \ref{epidemic_process}. Further, we define a multi-type super-critical branching process tailored to the epidemic process in Section \ref{Branching_process}. We describe the coupling between this and the epidemic process in Section \ref{coupling_defined}.  We then state the results demonstrating
 the closeness of the epidemic and the branching process in the early disease spread phase  in Section \ref{branching_results}. The results demonstrating the closeness
 of the epidemic process to its mean field limit once the epidemic process has grown are given in  in Section \ref{deterministic_results}.
The technical proofs are given separately in the Appendix. 


\subsection{Supercritical multi-type branching process review}
\label{define_standard_MTBP}
  In this section, we first define a multi-type branching process and state a key result associated with a super critical 
  multi-type branching process and an assumption outlining a set of sufficient conditions for it to hold.
  See \cite{branching_process_notes} for details.

Let ${\bm{\tilde{B}}_t}$ be a $\tilde{\eta}$ dimension vector denoting the multi-type branching process at time $t$, 
where $\tilde{\eta} < \infty$.  Component $i$ of ${\bm{\tilde{B}}_t}$ denotes  the number of individuals of type $i$ 
at time $t$. As is well known, in a multi-type branching process, at the end of each time period an individual may give birth to children of different types and itself dies (it may reincarnate as a child
 of same or different type).  
 %The probability that an individual of type $i$ has $r_j \in \mathbb{R^+} $ children of type $j$ in one time period is denoted by $p^i(r_1,...,r_{\eta})$. 
 The number of children each individual of type $i$ gives birth to  is independent and identically distributed. Therefore, multi-type branching process is a Markov chain $(\bm{\tilde{B}}_t \in \mathbb{{R^+}^{\tilde{\eta}}} :t \geq 0)$. Suppose  $\bm{\tilde{B}}_t = (b_1,\ldots,b_{\tilde{\eta}})$, then $\bm{\tilde{B}}_{t+1}$ is sum  of independent offsprings of $b_1$ type $1$ parents, independent offsprings $b_2$  type $2$ parents, and so on. Thus, $\bm{\tilde{B}}_{t+1}$ is sum of $b_1+b_2+...+ b_{\tilde{\eta}}$ independent random vectors in $\mathbb{{R^+}^{\tilde{\eta}}}$. 
 
% Let, $\bm{e_i} \in \mathbb{{R^+}^\eta}$ be a $\eta$ dimension vector whose $i^{th}$ component is 1 and other components are 0. If $\bm{B}_0 = \bm{e_i}$, then $\bm{B}_1$ will have following generating function :

%\[f^i(s_1,...,s_\eta) = \sum_{r_1,...,r_\eta=0}^{\infty}p^i(r_1,...,r_\eta)s_1^{r_1}...s_{\eta}^{r_\eta}\]

Consider a matrix  $\tilde{K}\in {\Re^+}^{\tilde{\eta} \times \tilde{\eta}}$ such that $\tilde{K}(i,j) = \Exp{\tilde{B}_1(j)|\bm{\tilde{B}}_0 =\bm{e_i}}$, that is expected number of type $j$ offsprings of a single type  $i$ individual in one time period. Then, 
\begin{equation*}
    \begin{aligned}
        \Exp{\bm{\tilde{B}}_{t+1}} &=  \tilde{K}^T \Exp{ \bm{\tilde{B}}_t } = (\tilde{K}^T)^{t+1} \Exp{\bm{\tilde{B}}_0}. 
    \end{aligned}
\end{equation*}


\noindent 
Recall that the Perron Frobenius eigenvalue of a non-negative irreducible matrix (a non-negative square matrix $A$ is irreducible if there exists an integer $m>0$ such that
all entries of $A^m$ are strictly positive) is its largest eigenvalue in absolute value, and can be seen to be positive. 
The following assumption is standard in multi-type branching process theory (see \cite{branching_process_notes}).
\begin{assumption} 
\label{Assumption_branching_standard}
     $\tilde{K}$ is  irreducible  and its Perron Frobenius eigenvalue $\tilde{\rho} >1.$   Furthermore, 
   \[
   \Exp{\tilde{B}_1(j)\log \tilde{B}_1(j)|\bm{\tilde{B}}_0=\bm{e_i}}<\infty,
   \]
    for all $ 1\leq i,j\leq \tilde{\eta}$. 
     %\footnote{A matrix $K\in {\Re^+}^{\eta \times \eta}$ is reducible if there exists a permutation matrix $P\in {\Re^+}^{\eta \times \eta}$, matrices $\tilde{K}\in {\Re^+}^{\tilde{\eta} \times \tilde{\eta}}$ $C\in {\Re^+}^{(\eta-\tilde{\eta}) \times (\eta - \tilde{\eta}})$ and $M\in {\Re^+}^{\eta \times (\eta - \tilde{\eta}})$ such that  $PKP^T=\big(\begin{smallmatrix} \tilde{K} & M \\ 0 & C\end{smallmatrix}\big)$. Otherwise it is irreducible.} such that $K(i,j)\geq0$ for all $1\leq i,j\leq\eta$ and strictly positive for some power.
 
     %Spectral radius $\rho$ of matrix $K$ that is the highest absolute value of its eigenvalues, $\rho > 1$ 
 \end{assumption}
 
  By $X_n \xrightarrow{P}X$ we denote  that the sequence of random variables $\{X_n\}$ converges to 
 $X$  in probability as $n \rightarrow \infty$.  Theorem~\ref{stbp_theorem} below   is well known (see \cite{branching_process_notes}).

 Suppose that Assumption \ref{Assumption_branching_standard} holds. Then, 
 corresponding to eigenvalue $\tilde{\rho}$, there exist  strictly positive right and left eigenvectors of $\tilde{K}$, $\bm{\tilde{u}}$ and $\bm{\tilde{v}}$   such that $\bm{\tilde{u}}^T\bm{\tilde{v}}=1$ and $\sum_{i=1}^{ \tilde{\eta}}{\tilde{u}(i)}  = 1$. 
 
\begin{theorem}
Under Assumption \ref{Assumption_branching_standard},
\[
\lim_{t \to \infty} \frac{\tilde{K}^t}{\tilde{\rho}^t} = \bm{\tilde{u}}\bm{\tilde{v}}^T.
\]
Furthermore, 
      \begin{equation} \label{eq:0001}
       \frac{\bm{\tilde{B}}_t}{\tilde{\rho}^t} \xrightarrow{P} W\bm{\tilde{v}} \quad \text{ as } t \to \infty,
       \end{equation} 
      where $W$ is a non-negative random variable such that $P\{W>0\}>0$ and $\Exp{W|\bm{\tilde{B}}_0=\bm{e_i}} =\tilde{u}(i)$ for all $i=1,...,\eta$. 
      Let $A=\{\omega:\tilde{B}_t(\omega) \to \infty \mbox{ as }t \to \infty\}$. Then, for any $\epsilon > 0$ and for all $j\in[1, \tilde{\eta}]$
  
 \begin{equation}\lim_{t\to \infty} P \{ \omega : \omega \in A, \abs{\frac{{\tilde{B}}_t(j)}{\sum_{i=1}^{ \tilde{\eta}}\tilde{B}_t(i)}-\frac{{\tilde{v}(j)}}{\sum_{i=1}^{ \tilde{\eta}}\tilde{v}(i)}}>\epsilon \}=0.
 \label{proportion_stbp}
 \end{equation}
  \label{stbp_theorem}
\end{theorem}

%Briefly, the epidemic process is as follows: city comprises of $N$ individuals with fraction $\pi_a$ in age group $a$ with a single `community' interaction space. An individual disease state can be susceptible, exposed, infected, symptomatic, hospitalised, critical, dead or recovered. The total number of contacts any infectious individual makes with all the individuals in the city in a time step is Poisson distributed with rate $\beta \Delta t$. Each contact is with an  individual chosen uniformly at random from $N$ individuals and if the contacted individual is susceptible, it becomes exposed, otherwise the contact has no effect.  Once an individual gets exposed, its disease progression is independent of all the other individuals in the city and depends only on its age. The time spent in each state (except susceptible, dead, recovered) is geometrically distributed. Transition to symptomatic (hospitalised, critical, dead) from respective earlier disease states happens with respective age dependent transition probabilities, otherwise the person recovers. Let $X_t^N \in \mathbbm{Z^+}^d$,  denote the number of individuals of each type (differing in age and disease state) excluding the susceptible population at time $t$.  

\subsection{Epidemic process dynamics}
\label{epidemic_process}
Some notation is needed to help specify the dynamics of the epidemic process.
   \begin{itemize}
       \item 
       The city comprises of $N$ individuals and our interest is in analyzing the city asymptotically as $N \rightarrow \infty$.
       %For analysis purposes, we consider separate epidemic processes defined on a common probability space and indexed by $N$ (the $N$ th process is also referred to as epidemic system  $N$).  
       \item
       An individual at any time can be in one of the following disease states: Susceptible, exposed, infective, symptomatic, hospitalised, critical, dead and recovered.  Individuals are infectious only in infective or symptomatic states. Denote all the disease states  by $\mathcal D$. To keep the discussion
       simple, we ignore the possibility of an individual getting reinfected, although incorporating them in a realistic way would not alter our conclusions.
       \item
       Each individual has some characteristics that are assumed to remain unchanged throughout the epidemic regardless of individual's  disease state. These include individual's age group, disease progression profile (e.g., some may be more infectious than others), 
       community transmission rates (e.g., individuals living in congested slums may be modelled to have higher transmission rates), 
       mobility in the community (e.g., elder population may travel less to the community compared to the working age population).  We 
       assume that set of all possible individual characteristics are finite, and  denote them by $\mathcal A$.
        Let $N_{a}$,  for $a \in \mathcal A$, denote the total number of individuals  with  characteristic $a$ in system $N$,
        and set  $ \pi_{a} =  \frac{N_{a}}{N}$ for all $a\in \mathcal A$. We assume that $\pi_{a}$ is independent of $N$ as $N \rightarrow \infty$.
     
       \item Hence,  each individual at any time may be classified by a  type $\bm{s} =(a,d)$, where $a$ denotes 
       the individual characteristic and $d$ the disease state. Let $\mathcal S = \mathcal A \times \mathcal D$
       denote the set of all types.  
       \item
Let $\mathcal U \subset \mathcal S$  denote all the \textit{types}  with susceptible disease state. Hence, $\mathcal S \setminus \mathcal U$ denote the types where individuals are already affected (that is, they have been exposed to the disease at some point in the past). Let $\eta = \lvert \mathcal S \setminus \mathcal U \rvert$.      
        
\item          
Denote the number of individuals of type $\bm{s}$ at time $t$ by  $X_t^N(\bm{s})$ and set 
$\bm{X}_t^N =( X_t^N(\bm{s}): \bm{s} \in  \mathcal S \setminus \mathcal U  )$. Then, ${\bm{X}_t^N \in \mathbbm{Z^+}^\eta}$.  
%\item 
%${\bm{\tilde{X}}_t} = (\tilde{X}_t(\bm{s}): \bm{s} is \text{exposed, infected, symptomatic})$  so that $\tilde{X}_t(\bm{s}) = X_t(\bm{s})$ for all $ \bm{s} \in \mathcal A$ .
 \item
 Let $A_t^N = \sum_{\bm{s}\in\mathcal S \setminus \mathcal U  } X_t^N(\bm{s})$ denote the  total number of affected individuals in the 
 system $N$ at or before time $t$.
\end{itemize}
    

\noindent \textbf{Dynamics:} At time zero, for each $N$,  ${\bm{X}_0^N}$ is initialised by setting a suitably selected small and fixed  number of people 
randomly from some distribution $\mu_0(N)$ and assigning them 
 to the exposed state. All others  are set as susceptible. 
 The distribution $\mu_0(N)$ is assumed to be independent of $N$ so we can set ${\bm{X}_0^N} = \bm{X}_0$  for all N. 
 
 Given ${\bm{X}_t^N}$, ${\bm{X}_{t+\Delta t}^N}$ is  arrived at  through two mechanisms.  For the ease of notation we will set $\Delta t =1$.   
1) Infectious individuals at time $t$ who make Poisson distributed infectious contacts with 
 the rest of the population, moving the contacted susceptible population to exposed state, and 2) through population already affected moving further
 along in their disease state. Specifically, 
 \begin{itemize}
 \item 
        We assume that an infectious individual with characteristic $a \in \mathcal A$ spreads the disease in the community with transmission rate $\beta_a$. 
        Thus, the total number of infectious contacts it makes  with all the individuals (both susceptible and affected)  in one time step is
         Poisson distributed with rate ${\beta_{a}}$. The individuals contacted are selected randomly and an individual with characteristic $\tilde{a}\in \mathcal A$ is selected with probability
         proportional to ${\psi_{a,\tilde{a}}}$ (independent of $N$). $\psi_{a,\tilde{a}}$  helps model biases such as an individual living in a dense region is more likely to infect another
         individual living in the same dense region.  As a normalisation, set $\sum_{\tilde{a}}\pi_{\tilde{a}} \psi_{a,\tilde{a}}=1$. 
         
         \item Once the number of infectious contacts for a particular characteristic $\tilde{a} \in \mathcal A$   have been generated, each contact is made with an individual selected uniformly at random from all the individuals with characteristic $\tilde{a}$.
         
         \item
          If an already affected individual has one or more contact 
         with an infectious individual, its type remains unchanged. On the other hand, 
      if a susceptible individual has at least one contact from any infectious individual, it will get exposed at time $t+1$.
      Each susceptible individual with characteristic $\tilde{a} \in \mathcal A$ who gets exposed
      at time $t$, increments 
      ${\bm{X}_{t+1}^N}(\tilde{a},exposed)$  by 1. A susceptible individual  that has no contact with an infectious individual remains susceptible
      in the next time period.         
       \item
        Once an individual gets exposed, its disease progression is independent of all the other individuals and depends only on its characteristics, that is, the disease progression profile of the characteristic class the individual belongs to. The waiting time in each state (except susceptible, dead and recovered) is assumed to be geometrically distributed. Transition to symptomatic, hospitalised, critical, dead or recovered state happens with respective characteristic (disease progression profile) dependent transition probabilities.  Thus, an individual of type $\bm{s} \in \mathcal S \setminus \mathcal U$, some disease state other than susceptible, at time $t$ transitions to some other state $\bm{q}$ at time $t+1$ with the transition probability $P(\bm{s},\bm{q})$ in one time step. The probability $P(\bm{s},\bm{q})$ is independent of time $t$
        and $N$.  If the transition happens,  $X_{t+1}^N(\bm{s})$ is decreased by 1 and $X_{t+1}^N(\bm{q})$ is increased by 1. Observe that, if characteristic of types $\bm{s}$ and $\bm{q}$ is different, then $P(\bm{s},\bm{q}) = 0$. 
         \end{itemize}
 
 %Above, we have specified a recipe for constructing  $({\bf X}_t^N: t \geq 0)$ for each $N \geq 1$.

 %Let $\mathcal S$ denote the set of all possible \textbf{``types''} of individuals $\bm{s}=(a,d)$ in the city considering their group of the individual and its disease state. 
       %All possible states denote types.  
 

      
       %The total number of contacts any infected individual (type $s\in \mathcal S$) makes with all the individuals of a characteristic (represented by some type $\bm q$)  in city in one time step is Poisson distributed with rate $\psi_{s,q}{\beta_{\bm s}} \Delta t$. Each contact is with an  individual chosen uniformly at random from $N$ individuals and if the contacted individual is susceptible, it becomes exposed otherwise contact has no effect. 
       
       
       %and with all the individuals with same characteristic (represented by some type ${\bm {q}}$) in one time step is Poisson distributed with rate ${\pi_{{\bm{q}}} \beta_{\bm{s}}}$.
       
        %Each contact is with an  individual chosen uniformly at random from $N$ individuals and if the contacted individual is susceptible, it becomes exposed otherwise contact has no effect. For the ease of notation we will set $\Delta t =1$.
        
        
       
       %Observe that number of contacts any infected individual (type $s\in \mathcal S$) makes with an individual in city in one time step is Poisson distributed with rate ${\frac{\beta_{\bm s}}{N}}$ and with all the individuals with same characteristic (represented by some type ${\bm {q}}$) in one time step is Poisson distributed with rate ${\pi_{{\bm{q}}} \beta_{\bm{s}}}$.




  %As in the epidemic process, each infectious individual gives birth to  Poisson distributed exposed individuals of each age group $a$  with rate $\pi_a \beta \Delta t$.  Once an exposed individual is born, disease progression of the individual has same probabilistic evolution as in epidemic process when they are coupled.  ${\bm{B}_t} \in \mathbbm{Z^+}^d$  denote the number of individuals of different types at time $t$ in the branching process.
 




\subsection{Associated branching process dynamics}
\label{Branching_process}


For each $t$, let ${{\bm{B}_t} \in \mathbbm{Z^+}^\eta}$, ${\bm{B}_t} =( B_t(\bm{s}):\bm{s} \in \mathcal S \setminus \mathcal U )$
where  $B_t(\bm{s})$ denote the number of individuals of $\bm{s}$ at time $t$ in the branching process.

%We  define $\{{\bm{B}_t}\}$ on the same probability space as $\{{\bm{X}_t^N}\}$ for each $N$, although we firstdiscuss the dynamics of $\{{\bm{B}_t}\}$, and discuss the details of the coupling of the these processes in Section ~\ref{coupling_defined}.  

 %   \item 
 %   ${\bm \tilde{B}_t} =( \tilde{B}_t(\bm{s}) : \bm{s} \in  \mathcal A)$
%where $\tilde{B}_t(\bm{s})= {B}_t(\bm{s})$ for all $ \bm{s} \in  \mathcal {A}$

\noindent \textbf{Dynamics:} At time zero ${\bm{B}_0} = {\bm{X}_0}$. Given ${\bm {B}_t}$, we arrive at ${\bm {B}_{t+1}}$ as follows: 
 \begin{itemize}
     \item
At time $t$, every infectious individual of type $a$, for all $a$,  gives birth to independent Poisson distributed 
offspring of   type $(\tilde{a},exposed)$ at time $t+1$  with rate $\pi_{\tilde{a}}\psi_{a, \tilde{a}} \beta_{a}$
for each $\tilde{a}$. $B_{t+1}(\tilde{a},exposed)$ is increased accordingly.
\item
Once an individual gets exposed, disease 
progression of the individual has same transition probabilities as in each epidemic process.   An individual of type
 $\bm{s} \in \mathcal S \setminus \mathcal U$, that is, in disease state other than susceptible at time $t$, transitions to some other disease state $\bm{q}$ at time $t+1$ with probability $P(\bm{s},\bm{q})$. If the transition happens, $B_{t+1}(\bm{s})$ is decreased by 1 and $B_{t+1}(\bm{q})$ is increased by 1.     
 \end{itemize}
 
 Let $A_t^B = \sum_{\bm{s}\in\mathcal S \setminus \mathcal U  } B_t(\bm{s})$ denote the  total number of offsprings generated
   by time $t$ in the branching process.  

 
 
\noindent \textbf{Specifying the expected offsprings matrix  $K\in {\Re^+}^{\eta \times \eta}$ for $\{{\bm{B}_t}\}$:} Let
each entry  ${K}(\bm{s},\bm{q})$ of $K$ denote the expected number of type $\bm{q}$  offspring of a single type $\bm{s}$ individual in one time step.
 Let $\mathcal{H} \subset \mathcal S$  denote all the types with the disease states that are infectious or may become infectious in subsequent time steps (that is, types with disease state either exposed, infective or symptomatic). Let $\mathcal H^c$ denote its complement. Individuals in
    $\mathcal H^c$ do not contribute to community infection.   Let ${\mathcal{\tilde{H}}} \subset \mathcal S$  denote all the types with  the disease states that are infectious (that is infective or symptomatic state). Let $\mathcal{E}$  denote $\mathcal{H}\setminus\mathcal{\tilde{H}}$, that  is, the set of all the types corresponding to exposed individuals.  Let $\hat{\eta} = \abs{\mathcal H}$. 

%Let $K \in {\Re^+}^{\eta \times \eta}$ be a matrix, where ${K}(\bm{s},\bm{q})$ is the expected number of type $\bm{q} = (\tilde{a},\tilde{d}) $ offspring of a single  type $\bm{s} = (a,d)$  individual (by itself transitioning to state $\bm{q}$ or giving birth to new exposed individuals in state $\bm{q}$) in one time step. 
 
 
    
As described above, an individual of type $\bm{s}\in \mathcal S \setminus \mathcal U$,  may give birth to other exposed individuals if it is infectious, and/or 
 may itself transition to some other type in one time step. Then, $K$ can be written as,

\begin {equation}
\begin{aligned}
K(\bm{s},\bm{q}) &= P(\bm{s},\bm{q}) + \pi_{\tilde{a}} \psi_{a, \tilde{a}} \beta_{a}  \text{ for all } \bm{s} = (a,d) \in \mathcal{\tilde{H}} \text{ and }
\bm{q} = (\tilde{a},\tilde{d}) \in\mathcal{E}
\\
 K(\bm{s},\bm{q}) &= P(\bm{s},\bm{q}) \text { otherwise.}  
\end{aligned}
\label{K_definition}
\end{equation}

 
% \noindent  As matrix $K$ is such that ${K}(\bm{s},\bm{q})$ is the expected number of type $\bm{q}$  offspring of an individual of type $\bm{s}$ (by itself transitioning to state $\bm{q}$ or giving birth to new exposed individuals in state $\bm{q}$) in one time step. 
It follows that
\begin{equation}
    \begin{aligned}
        \Exp{\bm{B}_{t+1}} &=  K^T \Exp{ \bm{B}_t }
        \\ & = (K^T)^{t+1} \Exp{ \bm{B}_0 }.
    \end{aligned}
    \label{branching_basic_eq}
\end{equation}
 
 
Recall that  Theorem \ref{stbp_theorem} holds for a standard multi-type branching process under Assumption \ref{Assumption_branching_standard}.  In particular, $\tilde{K}$ is assumed to be irreducible. 
  However, $K$  defined above is not irreducible. 
 Lemma \ref{matrix_structure} below sheds further light on $K$. 
  Theorem \ref{theorem_extending_branching_process} observes that conclusions of  Theorem \ref{stbp_theorem}
 continue to hold for the  branching process $\{{\bm{B}_t}\}$ associated with each epidemic process  $\{{\bm{X}_t^N}\}$.
 For any matrix $M$, let $\rho(M)$ denote its spectral radius, that is, the maximum of the absolute values of all its eigenvalues.
 
 Henceforth, we assume that $\rho =\rho(K) > 1$ in all subsequent analysis. 

 
\begin{lemma} 
 There exist matrices $K_1 \in {\Re^+}^{\hat\eta\times\hat\eta}$,  $C\in{\Re^+}^{(\eta-\hat\eta)\times(\eta-\hat\eta)} $ and $M \in {\Re^+}^{(\eta)\times(\eta-\hat\eta)}$ such that 

\[K=\big(\begin{smallmatrix} K_1 & M \\ 0 & C\end{smallmatrix}\big),\]

where $K_1 \in {\Re^+}^{\hat\eta\times\hat\eta}$  is  irreducible.
Further, $\rho(C)  \leq \rho(K_1)$. 
\label{matrix_structure}
\end{lemma}

%Below we show that this matrix $K$ has a special structure for which $\frac{K^t}{\rho^t} = uv^T$ is still true. 
 

 
% \[K=\big(\begin{smallmatrix} \tilde{K} & M \\ 0 & C\end{smallmatrix}\big)\]

%with matrix C having all its eigenvalues positive and less than equal to 1.


  %In Theorem \ref{theorem_extending_branching_process}, we will show that $\rho(K)$ (spectral radius of $K$) is an eigenvalue of $K$. Let  For ease of notation denote $\rho(K)$ by $\rho$.
 As $K_1$ defined in Lemma \ref{matrix_structure} is irreducible, its Perron Frobenius eigenvalue is equal to the spectral radius $\rho(K_1)$.
 
  \begin{theorem}
 Spectral radius of $K$ is equal to the spectral radius of $K_1$, that is, $\rho(K) =\rho(K_1)$. Furthermore, $\rho= \rho(K)$ is a unique eigenvalue of $K$ with maximum absolute value and 
\[ \lim_{t \to \infty} \frac{K^t}{\rho^t} = \bm{u}\bm{v}^T,\]
where $\bm{u}$ and $\bm{v}$ are the strictly positive right and left eigenvectors of $K$ corresponding to eigenvalue $\rho=\rho(K)$ such that $\bm{u}^T\bm{v}=1$ and $\sum_{i=1}^{\eta}{u}(i) = 1$. In addition,        
      \begin{equation*}
       \frac{\bm{B}_t}{\rho^t} \xrightarrow{P} W\bm{v} \quad \text{ as } t \to \infty,
       %\label{base_result_extbp}
       \end{equation*} 
      where $W$ is a non-negative random variable such that $P\{W>0\}>0$ iff $B_0(\bm{s})\neq0$ for some $\bm{s} \in \mathcal H$ and $\Exp{W|\bm{B}_0=\bm{e_i}} = u(i)$ for all $i=1,...,\eta$. Also, 
  let $A=\{\omega:B_t(\omega) \to \infty\}$ as $t \to \infty$. Then, for any $\epsilon > 0$ and for all $j\in[1,\eta]$,
  \begin{equation*}\lim_{t\to \infty} P \{ \omega : \omega \in A, \abs{\frac{{B}_t(j)}{\sum_{i=1}^{\eta}B_t(i)}-\frac{{v}(j)}{\sum_{i=1}^{\eta}v(i)}}>\epsilon \}=0.
  %\label{proportion_extbp}
  \end{equation*} 
   \label{theorem_extending_branching_process}  
 \end{theorem}
 
Observe that  
 $
   \Exp{{B}_1(j)\log {B}_1(j)|\bm{{B}}_0=\bm{e_i}}<\infty 
   $
    for all $ 1\leq i,j\leq {\eta}$ holds  because the offsprings for our branching process have a Poisson distribution.
    

 

 
 


%As a result, we can apply the multi-type branching processes results 1-3 of Section \ref{Standard_MTBP_results} to the branching process corresponding to epidemic process. 
%the matrix results (Section \ref{matrix_results}) helpful in proving them. These results will be later extended to the branching process corresponding to the epidemic process in Section \ref{proof_proposition_1}.


    

   
\subsection{Coupling the epidemic processes  and associated branching process}
\label{coupling_defined}
\begin{itemize}
\item
Recall that at time zero, $\bm{B}_0=\bm{X}_0$. We couple each exposed individual of each type in the epidemic process to an exposed individual of the same type in the branching process at time zero. 
\item
%As we see below, when an individual in system  $N_1$ is coupled to a branching process, a corresponding indidual  is coupled to the branching process in  all systems $N \geq N_1$. 
The coupled individuals in each of these processes follow the same disease progression (using same randomness)  and stay coupled throughout the simulation.
\item
Further,  when infectious, they generate  identical Poisson number of contacts (in epidemic process) and offsprings (in branching process).
Specifically,  when a coupled individual with characteristic $a\in \mathcal A$ is infectious, the number of contacts it makes in a time step 
 with all the individuals with characteristic $\tilde{a} \in \mathcal A$ in epidemic process is Poisson distributed with rate 
 $\pi_{\tilde{a}} \psi_{a,\tilde{a}} \beta_{a}$. This equals the offsprings generated by the
 corresponding coupled individual in the branching process where the  offsprings are  with 
 characteristic $\tilde{a} \in \mathcal A$ and are of  type $(\tilde{a},exposed)$.
 \item
 In epidemic process, each contact is made with an individual randomly selected from the population with the same characteristic. 
If a contact is with a susceptible person, then that person is marked exposed in the next time period
and is coupled with the corresponding person in the branching process.
%The epidemic systems for each $N$ can be coupled so that if a contact is with a susceptible person in system $N_1$, the it is with a susceptible person for all systems $N \geq N_1$. 
\item
 On the other hand, if in epidemic  process, a new contact is with an already  affected individual, then this does not result in any person getting exposed
so that the corresponding offspring in the branching process is uncoupled.  
%Again, the epidemic processes can be coupled so that if such a contact occurs in system $N_1$, then it occurs for  for all systems $N \leq N_1$. 
The descendants of uncoupled individuals in the branching process are also uncoupled. 
In our analysis, we will show that such uncoupled individuals in the branching process vis-a-vis epidemic process are a negligible fraction of the coupled individuals 
till time  $\log_\rho N^*$ for large $N$, where recall that $N^*= N/\log N$ and $\log_\rho N^* = \log N^* /\log \rho$. 
\end{itemize}

Let $I_t^N$ denote the number of coupled individuals between epidemic process  and branching process at time $t$. Since all the affected individuals in epidemic process are coupled with some individual in branching process of the same type, $I_t^N = A_t^N$
%where $A_t^N$ is the number affected  in  the epidemic process $N$ by time $t$. 
and we have,
\[
X_t^N(\bm{s}) \leq B_t(\bm{s}) \quad \forall \bm{s}\in \mathcal S \setminus  \mathcal U.\]
Further,
\begin{equation}
    {A_t^N} \leq {A^B_t}.
    \label{total_affected_ineq}
    \end{equation}

The  new uncoupled  individuals  born from the coupled individuals in branching process at each time $t$ are referred to as  ``ghost individuals'' and are denoted
 by $G_t^N$. As mentioned earlier, these and their descendants  remain uncoupled to epidemic system $N$. Let $D_{i,j}^N$ denote all the descendants of $G_i^N$ (ghost individuals born at time $i$) after $j$ time steps. Let $H_t^N$ denote the
 total number of uncoupled individuals in the branching process at time $t$. Then,
\begin{equation}H_t^N=
%\sum_{i=1}^t G_i^N \text{ and their descendants till time t} = 
\sum_{i=1}^t D_{i,t-i}^N.
\label{Ghost_eq}
\end{equation}

 

%We call these extra individuals 


 %As coupled individuals in an epidemic proce branching process at time $t-1$    

  %This can be assured because of the 2nd coupling in above paragraph



%\begin{enumerate}
 %\item
 %Both individuals are of the same age and are in same state at any time $t$ that is transition of a coupled individual in epidemic process is same as the corresponding coupled individual in branching process.
 %\item  
 %If the coupled individuals are infectious, the number of contacts made by a coupled individual with individuals in age group $a$ in epidemic process is equal to the number of exposed individuals of the age group $a$ corresponding coupled individual in branching process gives birth to (as both are Poisson distributed with rate $\beta$). 
 %\end{enumerate}
 
  %Let at time $t$, $I_t$ individuals each of the epidemic process and branching process are coupled. Individuals $I_{t+1}$ coupled at time $t+1$ are following:

%\begin{itemize}
 %   \item
  %  Individuals coupled at time $t$ will remain coupled till time $t+1$.
  %  \item
   
%\end{itemize}

%Through above coupling we ensure that, all the affected individuals in epidemic process at time $t$ are coupled. This statement is proved in Lemma 1.  

    \subsection{Analysis: The initial branching process phase}
\label{branching_results}
  %Broadly, the two processes only differ in that in the epidemic process contact with an already exposed person has no impact while in branching process this still gives birth to a new exposed individual uncoupled from the epidemic process.  Further, the branching process is independent of $N$. 

%The vector norm used in following results is the term by term absolute value of the vector, that is, $\left\lvert{{u}}\right\rvert (j) = \left\lvert{{u}(j)}\right\rvert $  for all $j\in[1,\eta]$.  
Following result shows that 
 epidemic process is close to the multi-type branching process till time $ \log_\rho N^*$ (recall that $N^*= N/\log N$ and $\log_\rho N^* = \log {N^*}/\log \rho$)
 as $N \rightarrow \infty$, where $\rho$ denotes the exponential growth rate of the branching process.


\begin{theorem} As $N\rightarrow\infty$, for all $\bm{s}\in \mathcal S \setminus  \mathcal U$,
        \begin{equation}
           \sup\limits_{t \in [0, \log_\rho ({N^*}/{\sqrt{N}})]} \left\lvert{X}_t^N(\bm{s})  - {B}_t(\bm{s}) \right\rvert \xrightarrow{P} 0.  
           \label{e_b_close_1}
        \end{equation}
        \begin{equation}
        \sup\limits_{t\in \left[0, \log_\rho N^*\right]} \left\lvert{\frac{{X}_t^N(\bm{s})}{\rho^t} - \frac{{B}_t(\bm{s})}{\rho^t}}\right\rvert \xrightarrow{P} 0.
        \label{e_b_close_2}
        \end{equation}
     \label{e_b_close}   
\end{theorem}
 
 Result (\ref{e_b_close_1}) was earlier shown for SIR setting in \cite{epidemic_notes}. We extend this result to general models and also prove the result (\ref{e_b_close_2}) in this more general setting. 

Following lemma is used in proving Theorem \ref{e_b_close}.
\begin{lemma} For all $\bm{s}\in \mathcal S \setminus  \mathcal U$,
\[\Exp{\abs{{{B}}_t (\bm{s}) - {{X}}^N_t(\bm{s})}} \leq \tilde{e} \frac{\rho^{2t}}{N},\]
for some constant $\tilde{e}>0$.
\label{bounding_diff_lemma_final}
\end{lemma}


Recall from Theorem \ref{theorem_extending_branching_process} that, 
\[ \frac{\bm{B}_t}{\rho^t} \xrightarrow{P} W\bm{v} \quad \text{ as } t \to \infty,\] 
 where $W$ is a non-negative random variable representing the intensity of branching process and ${\bm{v}} \in {\Re^+}^{\eta}$ is the left eigenvector corresponding to eigenvalue $\rho$ of matrix $K$.
Therefore, initially (till time $\log_{\rho}N^*$) epidemic process grows exponentially at rate $\rho$, with sample path dependent intensity being determined by $W$.

 The following proposition follows directly from Theorem \ref{theorem_extending_branching_process} and Theorem \ref{e_b_close} and justifies the fact that the proportions across different types stabilize quickly in the epidemic process, and thus early on,
 till time $\log_{\rho} N^*$, 
 paths can be patched from one time period
 to the other with negligible error due to change in the proportions. 
 
%Recall that, spectral value $\rho(K)$ of the matrix $K$ of branching process corresponding to the epidemic process is an eigenvalue of matrix $K$.  Let $\bm{v}$ be the corresponding left eigenvector.
 

 
\begin{proposition} For $t_N \rightarrow \infty$ as $N \rightarrow \infty$ and $\limsup_{N \rightarrow \infty}\frac{t_N}{\log_\rho N^*} 
< \infty$, then for all $\bm{s}\in \mathcal S \setminus  \mathcal U$: 
        \[ \left\lvert\frac{{X}_{t_N}^N(\bm{s})}{\sum_{\bm{q}\in \mathcal S \setminus  \mathcal U} {X}_{t_N}^N(\bm{q})} - \frac{{v}(\bm{s})}{\sum_{\bm{q}\in \mathcal S \setminus  \mathcal U} v(\bm{q})}\right\rvert \xrightarrow{P} 0, \quad \text{ as } ~N\rightarrow\infty. \]
        
\label{initial_proposition}
\end{proposition}
 
 \begin{remark}
 \em{In Algorithm \ref{alg:cap2} we had suggested that the restarted simulations should use common random numbers as in the original simulation
 paths so as to identically reproduce them. However, in our experiments we observe that even if the restarted paths are generated using independent  samples, that leads to a negligible anomaly. 
 To understand this,  observe that to replicate an original path, we essentially need to replicate $W$ along that path, as after small initial period, the associated branching process is well specified once $W$ is known. (Note that this $W$ is implicitly generated in a simulation, it is not explicitly computed). Common random numbers achieve this. 
 However, for approximating the statistics of the larger city after $t_{\min}$ (Day 22 in Figure \ref{shift_scale_restart_real_intervention}), 
 independent restarted simulations provide equally valid sample paths as the ones using the common random numbers. 
 The only difference is that the $W$ associated with each independently generated path may not match the $W$ associated with the original path,
 so patching them together at time $t_{\min}$ may result in a mismatch.  However, since we are reporting statistics associated with the average of generated paths, these statistics depend linearly on the average of the $W$'s associated with generated sample paths. Thus, if restarted paths are independently generated, the corresponding average of associated $W$'s  may not match 
the average of $W$'s associated with the original simulations. Empirically, we observe negligible mismatch 
as the average of $W$'s appear to have small variance. }
 \end{remark}
 
 Compartmental models are widely used to model epidemics. Usually, in these models, we start with infected population which is a positive fraction of the overall population. However, little is known about the dynamics  if we start with a small, constant number of infections. 
 Below we describe the results for some popular compartmental models using Theorem \ref{e_b_close} and \ref{initial_proposition}.



\begin{example}[SIR Model]{
Susceptible-Infectious-Recovered  (SIR) models are the simplest aggregate compartmental models
used in epidemiology. All the individuals in the city are assumed to be identical, i.e., they have same community transmission rate and disease progression transition probabilities. Individuals can be in three disease states (susceptible, infected, recovered). Let $S(t), I(t), R(t)$ denote the number of susceptible, infected and recovered at any time $t$. Let community transmission rate of an infected individual be $\beta$ and an infected individual recovers at rate $r$. SIR dynamics can be expressed by the following system of differential equations:

\[\frac{dS(t)}{dt} = - \frac{\beta}{N} S(t) I(t), \]
\[\frac{dI(t)}{dt} = \frac{\beta}{N} S(t) I(t) - r I(t),\]
\[\frac{dR(t)}{dt} =  r I(t).\]

A discrete version of this model fits our setting.  Again,  for the ease of notation, time step is set to 1. Let $\beta$ denote community transmission rate of an infected individual, i.e., number of contacts made by an individual with other individuals in one time step is Poisson distributed with rate $\beta$,  and an infected individual recovers at rate $r$.  Again, till time $t=\log_{\rho} N^*$, the epidemic process is close to the branching process. For branching process, let 
the infected population be denoted as type 1, and the recovered population as type 2. Then, the $K$ matrix of the corresponding 
branching process is:
\[K=\big(\begin{smallmatrix} 1+\beta -r  & r \\ 0 & 1\end{smallmatrix}\big).\]
%$P=\big(\begin{smallmatrix} 1  & 0 \\ 0 & 1\end{smallmatrix}\big)$ 
Observe that, matrix $K_1 = \big(1+\beta-r\big)$, $C=\big(1\big)$, $M=\big(r\big)$
%and right eigenvector $\tilde{u}=(1)$ and left eigenvector $\tilde{v}=(1)$  
and $\rho(K) = \rho(K_1) = 1+\beta-r$. Finally, 
%right eigenvector $u=\big(\begin{smallmatrix} 1 \\ 0\end{smallmatrix}\big)$ and left eigenvector 
\[v = \big(\begin{smallmatrix} 1 \\ \frac{r}{\beta-r}\end{smallmatrix}\big).\]
Therefore, initially (till time $\ \log_{\rho}N^*$ for large $N$) epidemic process grows exponentially at rate $\rho = 1+\beta-r$, and the ratio of infected to recovered individuals in this phase quickly stabilizes to $\frac{\beta-r}{r}$.
}
\end{example}






\begin{example}[SEIR model]
Recall that SEIR model corresponds to susceptible-exposed-infected-recovered model. As the name suggests, SEIR model considers an additional disease state of exposed as a refinement to SIR model. Again, all the individuals are identical. Community transmission rate of an infected individual is $\beta$ (i.e., number of contacts made by an individual with other individuals is Poisson distributed with rate $\beta$). An exposed individual transitions to infected state with rate $p$ and an infected individual recovers at rate $r$. 

Let the exposed population be denoted as type 1, infected population as type 2, and recovered population as type 3. Then, the  $K$ matrix of the branching process is:
\[K=\left ( \begin{matrix} 1-p & p & 0 \\ \beta & 1-r & r \\ 0 & 0 & 1 \end{matrix} \right ).\]
% $P=\big(\begin{smallmatrix} 1  & 0 & 0\\ 0 & 1 & 0 \\ 0 & 0 & 1\end{smallmatrix}\big)$  
Observe that, $C=\big(1\big)$, $M=\big(\begin{smallmatrix} 0 \\ r\end{smallmatrix}\big)$, $K_1 = \big(\begin{smallmatrix} 1-p  & p \\ \beta & 1-r\end{smallmatrix}\big)$ and $\rho(K) =\rho(K_1)=1+\frac{1}{2}\big(((p+r)^2+4p(\beta-r))^{1/2}-(p+r)\big)$.
%and right eigenvector $\tilde{u}=\frac{1}{2p+\rho-1}\big(\begin{smallmatrix}p  \\ p+\rho-1\end{smallmatrix}\big)$ and left eigenvector $\tilde{v}=\frac{2p+\rho-1}{p\beta+(p+\rho-1)^2}\big(\begin{smallmatrix} \beta \\ p+\rho-1\end{smallmatrix}\big)$ 
Finally, 
%\[u=\frac{1}{2p+\rho-1}\big(\begin{smallmatrix}p  \\ p+\rho-1 \\ 0 \end{smallmatrix}\big) \]
%\[v = \frac{2p+\rho-1}{p\beta+(p+\rho-1)^2}\big(\begin{smallmatrix} \beta \\ p+\rho-1 \\ r+\frac{rp}{\rho-1}\end{smallmatrix}\big)\]
\[v = \left (\begin{matrix} \beta \\ p+\rho-1 \\ r+\frac{rp}{\rho-1}\end{matrix} \right).\]

Therefore, initially (till time $t=\log_{\rho} N^*$, for large $N$) epidemic process grows exponentially at rate $\rho =1+\frac{1}{2}\big({((p+r)^2+4p(\beta-r))^{1/2}}-(p+r)\big)$, and the proportions  of exposed, infected and  recovered quickly stabilize to  
$\big(\beta, (p+\rho-1), (r+\frac{rp}{\rho-1})\big)$ normalised by their sum.
\end{example}

\begin{example}[SIR model with two age groups]
Usually different age groups have different disease progression profiles. To account for this, we consider SIR model with two age groups. 
We limit ourselves to two groups  so that the growth rate of epidemic process can be expressed in closed form (solution to a quadratic equation. In general, the degree of the equation is same as number of age groups). Suppose that the fraction of individuals in each age group is $\pi_1$ and $\pi_2=1-\pi_1$.
 There are only 3 disease states (susceptible, infected and recovered). Community transmission rate of an infected individual is $\beta$ (i.e.,  number of contacts made by an individual with other individuals is Poisson distributed with rate $\beta$). The infected individual make contacts with people belonging to age group 1 and age group 2 at rate $\pi_1 \beta$ and $\pi_2 \beta$, respectively. Further, suppose that an infected individual from first and second age group recover at rates  $r_1$ and $r_2$, respectively. 

We then have four types of population - First and second type denote the number infected in the first and second age group.  Third and fourth type
correspond to number recovered in first and second type age group.  Then the $K$ matrix of this branching process is:


\[K=\left(\begin{matrix} \pi_1 \beta + 1 - r_1 & \pi_2 \beta & r_1 & 0 \\ \pi_1 \beta & \pi_2 \beta + 1 - r_2 & 0 & r_2 \\ 0 & 0 & 1 & 0 \\ 0 & 0 & 0 & 1\end{matrix}\right ).\]

Here, as per the notation in Lemma 3.1,   $C=\big(\begin{smallmatrix} 1  & 0 \\ 0 & 1\end{smallmatrix}\big)$, $M=\big(\begin{smallmatrix} r_1  & 0 \\ 0 & r_2\end{smallmatrix}\big)$,
%$P=\big(\begin{smallmatrix} 1  & 0 & 0 & 0\\ 0 & 1 & 0 & 0\\ 0 & 0 & 1 & 0 \\ 0 & 0 & 0 & 1\end{smallmatrix}\big)$  
$K_1 = \left (\begin{matrix} \pi_1 \beta + 1 - r_1 & \pi_2 \beta  \\ \pi_1 \beta & \pi_2 \beta + 1 - r_2  \end{matrix} \right)$ and \[\rho(K) =\rho(K_1)=1+\frac{1}{2}\left(\left ((\beta-r_1-r_2)^2+4\pi_1r_2(\beta-r_1)+4\pi_2r_1(\beta-r_2) \right )^{1/2}+(\beta-r_1-r_2)\right)\]
%, right eigenvector $\tilde{u}=\frac{1}{1+\beta-r_1-\rho}\big(\begin{smallmatrix}\pi_2 \beta  \\ 1+\pi_1\beta-r_1\end{smallmatrix}\big)$ and left eigenvector $\tilde{v}=\frac{1+\beta-r_1-\rho}{\pi_1\pi_2\beta^2+(1+\beta-r_1-\rho)^2}\big(\begin{smallmatrix}\pi_1 \beta  \\ 1+\pi_1\beta-r_1\end{smallmatrix}\big).$
  

Finally, 
%right eigenvector $\frac{1}{1+\beta-r_1-\rho}\big(\begin{smallmatrix}\pi_2 \beta  \\ 1+\pi_1\beta-r_1 \\0 \\0 \end{smallmatrix}\big) $ and left eigenvector $v=P^T(\tilde{v},(M (\rho I-C)^{-1})^T \tilde{v})$, therefore 
%\[v = \frac{1+\beta-r_1-\rho}{\pi_1\pi_2\beta^2+(1+\beta-r_1-\rho)^2}\big(\begin{smallmatrix}\pi_1 \beta  \\ 1+\pi_1\beta-r_1 \\ \frac{r_1}{\rho-1} \pi_1 \beta  \\ \frac{r_2}{\rho-1} (1+\pi_1\beta-r_1) \end{smallmatrix}\big) \]
\[v = \left (\begin{matrix}\pi_1 \beta  \\ r_1+\rho - 1 - \pi_1\beta \\ \frac{r_1}{\rho-1} \pi_1 \beta  \\ \frac{r_2}{\rho-1} ( r_1+\rho - 1 - \pi_1\beta) \end{matrix} \right). \]
 Therefore, initially (for time $ \leq \log_{\rho}N^*$ and large $N$) epidemic process grows exponentially at rate 
 \[
 \rho =1+\frac{1}{2}\big( \left ((\beta-r_1-r_2)^2+4\pi_1r_2(\beta-r_1)+4\pi_2r_1(\beta-r_2) \right )^{1/2}+(\beta-r_1-r_2)\big),
 \]
  and the population of the four types (infected in two age groups and recovered in two age groups)
  quickly stabilizes to be proportional to $$\left(\pi_1 \beta, ( r_1+\rho - 1 - \pi_1\beta), (\frac{r_1}{\rho-1} \pi_1 \beta ), ( \frac{r_2}{\rho-1} ( r_1+\rho - 1 - \pi_1\beta)) \right).$$
\end{example}

\subsection{Deterministic phase}
\label{deterministic_results}

%Usually in SIR models, we start with infected population which is positive fraction of the overall population of the city, that is O(N) infected individuals. However, in our model we start with a small fixed number of infected individuals (for e.g., 100 individuals). In such a scenario, initially epidemic process grows like a branching process. Soon (in time $t=\alpha log_\rho mN$ ($0<\alpha<1$)), the ratio  of the infection population to recovered population stabilizes. The actual population infected, recovered is path dependent. On a given path, for two cities with population $N$ and $kN$, we have  $\frac{i_{N}(\alpha log_\rho N)}{i_{kN}(\alpha log_\rho kN)} = \frac{1}{k} $ and $\frac{r_{N}(\alpha log_\rho N)}{r_{kN}(\alpha log_\rho kN)} = \frac{1}{k} $. If we assume this to be true till we have O(mN) infections (time $t \approx \log_\rho (\epsilon m N)$, $0<\epsilon<1$), then $\frac{i_{N}(\log_\rho (\epsilon N))}{N}=\frac{i_{kN}(\log_\rho (\epsilon k N))}{kN}$ and $\frac{r_{N}(\log_\rho (\epsilon N))}{N}=\frac{r_{kN}(\log_\rho (\epsilon k N))}{kN}$. Therefore further evolution of two cities will be identical, that is for any $t>0$, $\frac{i_N (\log_\rho (\epsilon N)+t)}{N} = \frac{i_{kN}(\log_\rho (\epsilon kN)+t)}{kN}$ and $\frac{r_N (\log_\rho (\epsilon N)+t)}{N} = \frac{r_{kN}(\log_\rho (\epsilon k N)+t)}{kN}$.


%As Proposition 1 notes, early in the infection growth, the infection process proportions stabilize although the number infected  constitute a negligible fraction compared to the susceptible population. However, the actual

As Theorem \ref{e_b_close} and Proposition \ref{initial_proposition} note, early in the infection growth, 
till time $\log_\rho N^*$ for large $N$,
while the number affected grows exponentially, the proportion of individuals across different types stabilizes. Here, the types
corresponding to the susceptible population are not considered because at this stage, the affected are a negligible
fraction of the total susceptible population.  The growth in the affected population in this phase is sample path dependent and depends upon non-negative random variable  $W$. 

However, at time  $\log_{\rho} (\epsilon N)$ for any $\epsilon >0$ and large $N$, this changes as the affected population equals $\Theta(N)$. 
Hereafter, the population growth closely approximates its mean field limit whose initial state depends 
on random $W$ and where the proportions across types may change as the time progresses.
Our key result in this setting is Theorem~\ref{deterministic_theorem}. To this end we need Assumption~\ref{deterministic assumption}. 

Let $t_N= log_{\rho} (\epsilon N)$. Let $\bm{\mu}^N_{t}$ denote the empirical distribution across types at time $t_N+t$. This corresponds
to augmenting the vector $\bm{X}_{t_N+t}^N$  with the types associated with the susceptible population at time $t_N+t$
and scaling the resultant vector with factor $N^{-1}$.



\begin{assumption} 
There  exists  a random distribution  $\bm{\bar{\mu}}_{0}(W) \in {\Real^+}^{(\eta+1)}$ that is independent of $N$ 
such that  $\bm{\mu}^N_{0} \xrightarrow{P} \bm{\bar{\mu}}_{0}(W)$ as $N \rightarrow \infty$.
\label{deterministic assumption}
\end{assumption}

%{\bf Daksh, we have used $\mu^N_{0}$ earlier in the paper. Maybe their we should use different notation}

Observe that $\bm{\bar{\mu}}_{0}(W)$ above is path dependent in that  it depends on  the random variable $W$. 
%For ease of notation we will hide this dependence on $W$ and denote it by $\bm{\bar{\mu}}_0$.
The above assumption is seen to hold empirically. While, we do not have a proof for it (this appears to be a difficult and open problem),  Corollary \ref{e_b_close_epsilon} below
supports this assumption. This corollary follows from Lemma \ref{bounding_diff_lemma_final}.

%{\bf Daksh, as for Lemma \ref{bounding_diff_lemma_final}, improve the way this corollary is presented} \todo{Discuss}

\begin{corollary} For $\epsilon \in (0,1)$, $t=\log_\rho(\epsilon N)$ and for all $\bm{s}\in \mathcal S \setminus  \mathcal U$,
        \begin{equation*}
        \Exp{ \left\lvert{\frac{{X}_t^N(\bm{s})}{\rho^t} - \frac{{B}_t(\bm{s})}{\rho^t}}\right\rvert } \leq \epsilon \tilde{e},
        \end{equation*}
        for some constant $\tilde{e}>0$.
     \label{e_b_close_epsilon}   
\end{corollary}

%{\bf Daksh, please make the notation easier above. As we had discussed earlier.}


%{Daksh, lets discuss the below statement. We have not defined $\bm{X}_{t}^N$ and $\bm{X}_{t}^{kN}$ on the same space. Currently, these are a sequence of systems}.

%This will imply that, for $\epsilon \in (0,1)$, $t_1=\log_\rho(\epsilon N)$ and $t_2=\log_\rho(\epsilon k N)$, ($k>1$), we have 
   %     \begin{equation*}
     %    \Exp{\left\lvert{\frac{\bm{X}_{t_1}^N}{N} - \frac{\bm{X}_{t_2}^{kN}}{kN}}\right\rvert} \leq  2\epsilon^2 \tilde{e} \bm 1,
     %   \end{equation*}

 Let $c^N_{t}({a})$ denote the total incoming-infection rate from the community as seen by an individual with characteristic ${a}\in\mathcal A$ at time $t$.  It is determined from the disease state of all the individuals at time  $t$ and is $\mathcal F_{t}$ measurable. In our setup, $c^N_{t}({a})$ equals 
 
  \begin{equation}
    \sum\limits_{\bm{q}=(\tilde{a},\tilde{d}) \in \mathcal S \setminus  \mathcal U} \mu^N_{t}(\bm{q}) \mathbbm{1}(\text{type }\bm{q} \text{ is infectious}) \psi_{\tilde{a},{a}} \beta_{\tilde{a}},
    \label{cnt}
    \end{equation}
  
 % Let $g:(\mathcal S) \times (\mathcal A) \rightarrow \Re$ be  

% \[ g(\bm{q}, {a}) = \mathbbm{1}(\text{state }\bm{q}\text{ is infectious}) \psi_{\tilde{a},{a}} \beta_{\tilde{a}} ,\] 
 
% where $\bm{q}=(\tilde{a},\tilde{d})$. 
 
%  Clearly, $g$ is a bounded and continuous function. With this notation, $c^N_{t}({a})$, equals
%  $$\sum\limits_{\bm{q} \in \mathcal S} \mu^N_{t}(\bm q)g(\bm q,{a})$$ 
  
  
  
 
  
  where $\mathbbm{1}$ denotes the indicator function. 
  
  For each individual $n \leq N$, let $S^t_n$ denote its type at time $t$.  Then,  for  $\bm{s}=(a,d) \in \mathcal S$
  
  $$\mathbb{P}\left( S^{t+1}_n = \bm{s} | \mathcal F_{t} \right) = {h}(S^{t}_n, \bm{s},  c^N_{t}({a})),$$
 for a continuous function $h$. 
 In particular, the transition probability only depends on the disease-state of the individual at the previous time, the disease-state to which it is transitioning, and the infection rate incoming to the individual at that time. 
%$\tilde{h}(\bm{s'},\bm{s},\bar{c}_{t-1}({a}))$  can also be understood as the proportion of population of type $\bm{s'}$  that migrates to type $\bm{s} = ({a},{d})$ in one time step at time $t$.


% Define, a function $h$ such that $h(\bm{s'},\bm{s},\bar{\mu}_{t}) =  \tilde{h}(\bm{s'},\bm{s},\bar{c}_{t}({a}))$ for all $s'\in\mathcal S$,  $s=(a,d)\in\mathcal S$ and $\bar{c}_{t}({a}) = \sum\limits_{\bm{q} \in \mathcal S}\bar{\mu}_{t}(\bm q)g(\bm q, {a})$. \todo{Discuss: is this needed}
 
Recall that from Assumption \ref{deterministic assumption} we have defined $\bm{\bar{\mu}}_{0}(W) \in {\Real^+}^{(\eta+1)}$
such that  $\bm{\mu}^N_{0} \xrightarrow{P} \bm{\bar{\mu}}_{0}(W)$ as $N \rightarrow \infty$. Define  $\bm{\bar{\mu}}_{t}(W) \in {\Real^+}^{(\eta+1)}$ such that for all $t\in\mathbbm{N}$, $\bm{s}=(a,d) \in \mathcal S$,

\begin{equation}
        \label{mu_definition_equation}
     \bar{\mu}_{t}(\bm{s},W) := \sum\limits_{\bm{s'} \in \mathcal S} \bar{\mu}_{t-1}({\bm{s'}},W) h(\bm{s'}\bm{s},\bar{c}_{t-1}(a,W)), \end{equation}
     
     where \[\bar{c}_{t-1}({a},W) = \sum\limits_{\bm{q}=(\tilde{a},\tilde{d}) \in \mathcal S \setminus  \mathcal U} \bar{\mu}_{t-1}(\bm{q}, W)\mathbbm{1}(\text{state }\bm{q} \text{ is infectious}) \psi_{\tilde{a},{a}} \beta_{\tilde{a}}.\]



\begin{theorem}  Under Assumption \ref{deterministic assumption} and for $t\in\mathbbm N$, 
   \[
    \bm{\mu}^N_{t} \xrightarrow{P} \bm{\bar{\mu}}_{t}(W) \text{ as } N\rightarrow\infty. 
    \]
    
    
        %and $h(\bm{s'},\bm{s},\bar{\mu}_{t})$ is the proportion of population of type $\bm{s'}$  that migrates to type $\bm{s}$ in one time step.
        
        \label{deterministic_theorem}
        \end{theorem}
        
        %{\bf Daksh, $h$ needs more detail here. Lets discuss.}
        
        In particular, if $\bm{\bar{\mu}}_t$ denotes the mean field limit of the normalised process at time $t + \log_{\rho} (\epsilon N)$, then, the number of infections observed in a smaller model with population $N$ is approximately $N* \bm{\bar{\mu}}_t$ and that of a larger model is approximately $kN* \bm{\bar{\mu}}_t$. Thus, the larger model infection process can be approximated by the smaller model infection process by scaling it by $k$.
        
        \begin{remark} {\em
        While the deterministic equations (conditioned on $W$) in Theorem \ref{deterministic_theorem}  
        can be easily solved when the number of types is small and initial state is established via simulation, these become much harder in a realistic model with all its complexity, where the number of types is extremely large and may be uncountable if non-memoryless probability distributions are involved. 
  One may thus view the key  role of the simulator as a tool that identifies the random initial state of these deterministic mean field equations 
  and then solves them somewhat efficiently using stochastic methods.}
  \end{remark}
         
 %\begin{remark} {\em
% Another noteworthy observation is that while  epidemic process grows exponentially 
% as a branching process for $O(log_\rho N)$ time while once it enters the deterministic phase, it lasts for $O(1)$ time. }
  %\end{remark}
    


  
  
   
 
    
  










 




\section*{Acknowledgments}
We  acknowledge the support of A.T.E. Chandra Foundation for this research.
We further  acknowledge the support of the Department of Atomic Energy, Government of India, to TIFR under project no. 12-R\&D-TFR-5.01-0500. 



%\bibliographystyle{plain}
%{
%\bibliography{Mumbai_August_2}
%}


\begin{thebibliography}{10}

\bibitem{City_Simulator_IISc_TIFR_2020}
S.~Agrawal, S.~Bhandari, A.~Bhattacharjee, A.~Deo, N.M. Dixit, P.~Harsha,
  S.~Juneja, P.~Kesarwani, A.K. Swamy, P.~Patil, N.~Rathod, R.~Saptharishi,
  S.~Shriram, P.~Srivastava, R.~Sundaresan, N.~K. Vaidhiyan, and S.~Yasodharan.
\newblock City-scale agent-based simulators for the study of non-pharmaceutical
  interventions in the context of the covid-19 epidemic.
\newblock {\em Journal of the Indian Institute of Science}, 100(4):809--847,
  2020.

\bibitem{branching_process_notes}
K.B. Athreya and P.E. Ney.
\newblock {\em Branching processes}.
\newblock 1972.

\bibitem{SIR5}
O.~N. Bjørnstad, Katriona Shea, Martin Krzywinski, and Naomi Altman.
\newblock {The SEIRS model for infectious disease dynamics}.
\newblock {\em Nature}, 2020.

\bibitem{epidemic_notes}
T.~Britton and E.~Pardoux.
\newblock {\em Stochastic Epidemic Models with Inference}.
\newblock 2019.

\bibitem{SIR1}
Mohammadali Dashtbali and Mehdi Mirzaie.
\newblock {A compartmental model that predicts the effect of social distancing
  and vaccination on controlling COVID-19}.
\newblock {\em Nature}, 2021.

\bibitem{ferguson2005strategies}
Neil~M Ferguson, Derek~AT Cummings, Simon Cauchemez, Christophe Fraser, Steven
  Riley, Aronrag Meeyai, Sopon Iamsirithaworn, and Donald~S Burke.
\newblock Strategies for containing an emerging influenza pandemic in
  {Southeast Asia}.
\newblock {\em Nature}, 437(7056):209--214, 2005.

\bibitem{ferguson2020report}
Neil~M Ferguson, Daniel Laydon, Gemma Nedjati~Gilani, Natsuko Imai, Kylie
  Ainslie, Marc Baguelin, Sangeeta Bhatia, Adhiratha Boonyasiri, ZULMA
  Cucunuba~Perez, Gina Cuomo-Dannenburg, et~al.
\newblock {Impact of non-pharmaceutical interventions (NPIs) to reduce COVID19
  mortality and healthcare demand}.
\newblock Technical Report~9, MRC Centre for Global Infectious Disease
  Analysis, Imperial College London, U.K., 2020.
\newblock
  \url{https://www.imperial.ac.uk/mrc-global-infectious-disease-analysis/covid-19/report-9-impact-of-npis-on-covid-19/}.

\bibitem{SIR3}
J.~Fernández-Villaverdea and Charles Jones.
\newblock {Estimating and simulating a SIRD Model of COVID-19 for many
  countries, states, and cities}.
\newblock {\em Journal of Economic Dynamics and Control}, 2022.

\bibitem{gardner2020intervention}
Jasmine~M Gardner, Lander Willem, Wouter van~der Wijngaart, Shina Caroline~Lynn
  Kamerlin, Nele Brusselaers, and Peter Kasson.
\newblock Intervention strategies against covid-19 and their estimated impact
  on swedish healthcare capacity.
\newblock {\em medRxiv}, 2020.

\bibitem{halloran2008modeling}
M~Elizabeth Halloran, Neil~M Ferguson, Stephen Eubank, Ira~M Longini, Derek~AT
  Cummings, Bryan Lewis, Shufu Xu, Christophe Fraser, Anil Vullikanti,
  Timothy~C Germann, et~al.
\newblock {Modeling targeted layered containment of an influenza pandemic in
  the United States}.
\newblock {\em Proceedings of the National Academy of Sciences},
  105(12):4639--4644, 2008.

\bibitem{branching_process_notes_2}
T.E. Harris.
\newblock {\em The Theory of Branching Processes}.
\newblock 1963.

\bibitem{October_report_2020}
P.~Harsha, S.~Juneja, D.~Mittal, and R.~Saptharishi.
\newblock Covid-19 epidemic in {Mumbai}: Projections, full economic opening,
  and containment zones versus contact tracing and testing: An update.
\newblock Technical report, TIFR Mumbai, India, 2020.
\newblock \url{https://arxiv.org/abs/2011.02032}.

\bibitem{report2}
P.~Harsha, S.~Juneja, P.~Patil, N.~Rathod, R.~Saptharishi, A.Y. Sarath,
  S.~Sriram, P.~Srivastava, R.~Sundaresan, and N.K. Vaidhiyan.
\newblock {COVID-19 Epidemic Study II: Phased emergence from the lockdown in
  Mumbai}.
\newblock Technical report, India, 2020.
\newblock \url{https://arxiv.org/abs/2006.03375}.

\bibitem{harsha2020covidmumbai}
P.~Harsha, S.~Juneja, and R.~Saptharishi.
\newblock Covid-19 epidemic in mumbai: Long term projections, full economic
  opening, and containment zones versus contact tracing and testing.
\newblock Technical report, TIFR Mumbai, India, 2020.
\newblock
  \url{http://www.tcs.tifr.res.in/~sandeepj/avail_papers/Mumbai_September_Report.pdf}.

\bibitem{SIR4}
H.W. Hethcote.
\newblock The mathematics of infectious diseases.
\newblock {\em SIAM}, 2000.

\bibitem{hunter2017taxonomy}
Elizabeth Hunter, Brian Mac~Namee, and John~D Kelleher.
\newblock A taxonomy for agent-based models in human infectious disease
  epidemiology.
\newblock {\em Journal of Artificial Societies and Social Simulation}, 2017.

\bibitem{May_report_2021}
S.~Juneja and D.~Mittal.
\newblock {Modelling the Second Covid-19 Wave in Mumbai}.
\newblock Technical report, TIFR Mumbai, India, 2021.
\newblock \url{https://arxiv.org/pdf/2105.02144.pdf}.

\bibitem{kermack1927contribution}
William~Ogilvy Kermack and Anderson~G McKendrick.
\newblock A contribution to the mathematical theory of epidemics.
\newblock {\em Proceedings of the Royal Society of London. Series A, Containing
  papers of a mathematical and physical character}, 115(772):700--721, 1927.

\bibitem{malani2020serosurvey}
A.~Malani, D.~Shah, G.~Kang, G.N. Lobo, J.~Shastri, M.~Mohanan, R.~Jain, S.T.
  Agrawal, S.~Juneja, S.~Imad, and U.~Kolthur-Seetharam.
\newblock {Seroprevalence of SARS-CoV-2 in slums versus non-slums in Mumbai,
  India}.
\newblock {\em The Lancet Global Health}, 9(2):E110--E111, 2020.

\bibitem{matrix_analysis_notes}
M.W. Meckes.
\newblock {\em Lecture notes on matrix analysis}.
\newblock 2019.

\bibitem{sigmetricsposter}
D.~Mittal, S.~Juneja, and S.~Agrawal.
\newblock {Shift, scale and restart smaller models to estimate larger ones:
  Agent based simulators in epidemiology}.
\newblock {\em {To appear in ACM SIGMETRICS - Performance Evaluation Review}},
  2022.

\bibitem{verity2020estimates}
Robert Verity, Lucy~C Okell, Ilaria Dorigatti, Peter Winskill, Charles
  Whittaker, Natsuko Imai, Gina Cuomo-Dannenburg, Hayley Thompson, Patrick~GT
  Walker, Han Fu, et~al.
\newblock Estimates of the severity of coronavirus disease 2019: a model-based
  analysis.
\newblock {\em The Lancet Infectious Diseases}, 2020.

\end{thebibliography}

        

\section{Appendix}
\label{appendix}
In this section we first give proofs (Section \ref{Appendix 1}) of the results stated in Section \ref{Theoretical_results}. In Section \ref{heuristic_initial}, we provide a heuristic argument for the mismatch between larger model and smaller model when we start with small number of initial infections. Finally, we give details of the parameters and the city statistics used in numerical experiments in Section \ref{Numerical Parameters Section}.

\subsection{Proofs}

\label{Appendix 1}

\subsubsection{Proof of Lemma \ref{matrix_structure}}

\begin{itemize}


\item
Observe that
types $\bm{s} \in \mathcal{H}^c\setminus\mathcal U$ correspond to hospitalised, critical, dead or recovered states and types $\bm{q} \in \mathcal{H}$ correspond to exposed infective or symptomatic state. As any individual in hospitalised, critical, dead or recovered disease states neither transitions nor gives birth to offspring in exposed, infective or symptomatic disease states, therefore $K(\bm{s},\bm{q}) = 0 $ for all $\bm{s} \in  \mathcal{H}^c\setminus\mathcal U$ and  $\bm{q} \in \mathcal{H}$.

\item
Recall that $K_1 \in  {\Re^+}^{\hat{\eta} \times \hat{\eta}}$ is such that
$K_1(\bm{s},\bm{q}) =K(\bm{s},\bm{q}) $ for all $\bm{s} \in\mathcal{H}$ and $\bm{q} \in \mathcal{H}$. Observe that $K_1$ correspond to the ``types'' with the disease states that are infectious or may become infectious in subsequent time steps. As every infectious individual gives birth to exposed individuals of each characteristic therefore $K_1$ is irreducible. 


\item
Recall that matrix $C \in {\Re^+}^{(\eta - \hat{\eta}) \times (\eta - \hat{\eta})} $ is such that
$C(\bm{s},\bm{q}) =K(\bm{s},\bm{q}) $ for all $\bm{s} \in \mathcal{H}^c\setminus\mathcal U$ and $\bm{q} \in  \mathcal{H}^c\setminus\mathcal U$. The individuals of the type $\bm{s} \in \mathcal{H}^c\setminus\mathcal U$ do not contribute to any infections in the city. In addition, this individual either transitions to the next disease state or remains in the same disease state with positive probability. Therefore matrix $C$ is an upper triangular matrix with diagonal entries equalling the probability of an individual in some disease state remaining in the same disease state in one time step (which is less than 1).  Hence, all the eigenvalues of matrix C are positive and less than or equal to 1. 


\item
Recall that matrix $M \in {\Re^+}^{\hat{\eta} \times (\eta - \hat{\eta})} $ is such that
$M(\bm{s},\bm{q}) =K(\bm{s},\bm{q}) $ for all $\bm{s} \in \mathcal{H}$ and $\bm{q} \in\mathcal{H}^c\setminus\mathcal U$. 
\end{itemize}



 Therefore, there exists $K_1 \in {\Re^+}^{\hat\eta\times\hat\eta}$,  $C\in{\Re^+}^{(\eta-\hat\eta)\times(\eta-\hat\eta)} $ and $M \in {\Re^+}^{(\eta)\times(\eta-\hat\eta)}$ such that 

\[K=\big(\begin{smallmatrix} K_1 & M \\ 0 & C\end{smallmatrix}\big),\]

where $K_1 \in {\Re^+}^{\hat\eta\times\hat\eta} $ is irreducible,  and $\rho(C) < \rho(K)=\rho(K_1)$. 


%\label{matrix_results}
% Here we discuss some definitions and results regarding matrices (see \cite{matrix_analysis_notes} for more details), which are used in proving the standard multi-type branching process results. We will then extend these matrix results and standard multi-type branching process results in Section \ref{proof_proposition_1}.  Following \textbf{definitions} will help specify those results :


% \textbf{Result 1 :} 
%     Spectral radius $\rho$ of a strictly positive matrix $M$  is its eigenvalue with algebraic multiplicity 1 and absolute value of all other eigenvalues is strictly less than $\rho$. 

 \subsubsection{Proof of Theorem \ref{theorem_extending_branching_process}}




 
%\textbf{Result 3:}
% For a matrix $M$ satisfying $\lim_{t \to \infty} \frac{M^t}{\rho^t} = M_1$ with $\rho>1$, we can prove that  
% \begin{equation}
%     \begin{aligned}
%          M^t \le {\rho}^t  M_\delta \quad  \forall \quad  t\ge 0 
%          \label{matrix_bound}
%     \end{aligned}
% \end{equation}

%for some constant matrix $M_\delta$. As $\lim\limits_{i\rightarrow\infty} \lrp{\frac{{M}}{\rho}}^i = M_1 $, we have  
%    \begin{equation*}
%      \forall \delta > 0, \exists h_\delta \in \mathbb{N} \text{ such that } \forall j\ge h_\delta, \lrp{\frac{{M}}{\rho}}^j \le M_1+\delta \bm{1}\bm{1}^T,
%    \end{equation*}
 %   where $\bm{1}\bm{1}^T$ is the all ones matrix of the same dimension as $M$. Next, consider the matrix $D_{\delta}$ of the same dimension as $M$, whose $(i,j)$-th entry is given by 
 %   \[[M_{\delta}]_{ij} = \max\lrset{ I_{ij},M_{ij},[M^2]_{ij},\dots ,[M^{H_{\delta}}]_{ij},[ M_1+\delta \bm{1}\bm{1}^T]_{ij}}.\]

 
%\textbf{Note:} Clearly, Result 1 for positive matrices implies that Result 2 also holds for positive matrices. Results 1 and 2, also hold for non negative irreducible matrices which are strictly positive for some power.
 

%Consider a branching process for which assumption of non singularity is true but matrix $K$ is non negative (instead of being strictly positive) such that for some permutation matrix P,

% Matrix $K \in {\Re^+}^{\eta\times\eta} $, permutation matrix $P\in {\Re^+}^{\eta\times\eta}$,$\tilde{K} \in {\Re^+}^{\tilde\eta\times\tilde\eta} $, $C\in{\Re^+}^{(\eta-\tilde\eta)\times(\eta-\tilde\eta)} $ and $B \in {\Re^+}^{(\eta)\times(\eta-\tilde\eta)}$ such that 

%\[PKP^T=\big(\begin{smallmatrix} \tilde{K} & M \\ 0 & C\end{smallmatrix}\big)\]

%where $\tilde{K} \in {\Re^+}^{\tilde\eta\times\tilde\eta} $ is a irreducible matrix strictly positive for some power and $\rho(C) < \rho(\tilde{K})$. 

Recall from Lemma \ref{matrix_structure}, \[K=\big(\begin{smallmatrix} K_1 & M \\ 0 & C\end{smallmatrix}\big).\]
Observe that, set of eigenvalues of $K$ is same as combined set of eigenvalues of $K_1$ and $C$. Furthermore,
\begin{itemize}
    \item 
     $\rho(C) \leq 1$ from proof of Lemma \ref{matrix_structure},  therefore $\rho(K)=\rho(K_1)$.
    \item
    $K_1$ is irreducible from Lemma \ref{matrix_structure}, therefore its spectral radius $\rho(K_1)$ is Perron-Frobenius eigenvalue. In addition, spectral radius $\rho(K) = \rho(K_1)$ of $K$ is also a unique eigenvalue of $K$ with highest absolute value.
    \item
    $\lim_{t \to \infty} \frac{K^t}{\rho^t} = \bm{u}\bm{v}^T$ then follows from \cite{matrix_analysis_notes} (Lemma 8.16, pg. 69).
\end{itemize}




%As $K_1$ is an irreducible matrix strictly positive for some power, therefore $\rho(K_1)$ is an eigenvalue of $K_1$ with algebraic multiplicity \footnote{The algebraic multiplicity of an eigenvalue is the number of times it appears as a root of the characteristic polynomial.} 1 (Perron Frobenius Theorem).   Hence $\rho(K)$ is an eigenvalue of $K$ with algebraic multiplicity 1 (therefore, geometric multiplicity \footnote{The geometric multiplicity of an eigenvalue is the dimension of the linear space of its associated eigenvectors. Geometric multiplicity of an eigenvalue is always less then its algebraic multiplicity.} is also 1). Therefore, second part of Lemma \ref{general_matrix_result} follows from below result mentioned in \cite{matrix_analysis_notes} :  


 %Let there be a matrix $M$ and spectral radius $\rho$ of $M$ be its eigenvalue with geometric multiplicity 1 and it is the only eigenvalue of $M$ with absolute value $\rho$. Then,
%\[ \lim_{t \to \infty} \frac{M^t}{\rho^t} = \bm{u}\bm{v}^T\]
% where $\bm{u}$ and $\bm{v}$ are the right and left eigenvectors of matrix $M$ corresponding to eigenvalue $\rho$ such that $\bm{u}^T\bm{v}=1$ and $\bm{u}^T\bm{1} = 1$. 
 
%Observe that, eigenvectors for matrix $K$ corresponding to eigenvalue $\rho(K)$ are $\bm{u}=P^T(\bm{u_1},0)$ and $\bm{v}=P^T(\bm{v_1},(M (\rho I-C)^{-1})^T \bm{v_1})$, where $\bm{u_1},\bm{v_1}$ be the right and left eigenvectors of matrix $K_1$ corresponding to eigenvalue $\rho(K_1)$.
 




Results (\ref{eq:0001}) and (\ref{proportion_stbp}) in Theorem \ref{stbp_theorem} for supercritical multi-type branching process (Section \ref{define_standard_MTBP})
were proved in \cite{branching_process_notes}, Chap. V. Critical step in proving these results was the fact that corresponding matrix $\tilde{K}$ had following property $ \lim_{t \to \infty} \frac{\tilde{K}^t}{\tilde{\rho}^t} = \bm{\tilde{u}}\bm{\tilde{v}}^T$ (see Lemma 1 - Chap. V, pg. 194,  in \cite{branching_process_notes}).

As shown above, matrix $K$ of the associated branching process also satisfies this property ($ \lim_{t \to \infty} \frac{K^t}{\rho^t} = \bm{u}\bm{v}^T $), therefore results (\ref{eq:0001}) and (\ref{proportion_stbp}) in Theorem \ref{stbp_theorem} hold for our branching process as well.
%with a slight modification in Theorem \ref{stbp_theorem} to account for the fact that if the branching process starts from an individual of type $\bm{s} \in \mathcal {H}^c \setminus \mathcal U$,  a type which only transition itself to some other type and does not have multiple offspring than the branching process will never grow to infinity.


\subsubsection{Proof of Lemma \ref{bounding_diff_lemma_final}}

%\begin{lemma}

%For epidemic process and the corresponding branching process, we have 
%\[X_t^N(\bm{s}) \leq B_t(\bm{s}) \quad \forall \bm{s}\in \mathcal{S}\setminus %\mathcal V\]
%and therefore
%\[{A_t^N} \leq {A^B_t}\]


%\end{lemma}

%\begin{proof}

%Observe that if all the affected individuals in epidemic process are coupled with some individual in branching process at any time $t$, then the result $X_t^N(\bm{s}) \leq B_t(\bm{s})$ for all $\bm{s}\in \mathcal{S}\setminus \mathcal V$ and hence ${A_t^N} \leq {A^B_t}$ follows trivially. Therefor, it suffices to prove that all affected individuals in epidemic process are coupled with some individual in branching process at time $t$. We will prove this by induction:
%\begin{itemize}
%    \item 
 %   Base Case : As $\bm{B}_0=\bm{X}_0$, and we have coupled each infected individual in epidemic process with some individual in branching process, therefore claim is true at $t=0$.
  %  \item
  %  Hypothesis: Assume it to be true at time $t$, that is all affected individuals in epidemic process are coupled with some individual in branching process. 
  %  \item
  %  The susceptible person exposed at time $t+1$ through a contact of coupled individual in epidemic process will be coupled with an exposed individual born from the corresponding coupled individual in branching process (as number of contacts in epidemic process by a coupled individual is same as number of births by an the corresponding coupled individual in branching process). Also, all the coupled individuals at time $t$, remain coupled at time $t+1$.
    
   % Therefore all affected individuals in epidemic process are coupled with some individual in branching process at time $t+1$.
    
    
%\end{itemize}

%Therefore, at any time $t$ there will be at least as many individuals in state $\bm{s}$ in branching process as in epidemic process.

%\end{proof}



%Using bound \ref{affected_square_bound_final} in result from step 1, we get
Recall that total number of uncoupled individuals in branching process at time $t$ is $H^N_t$. Therefore, for all $\bm{s}\in \mathcal S \setminus  \mathcal U$
 \begin{equation}\abs{{{B}}_t(\bm{s}) - {{X}}^N_t(\bm{s})}  \le H^N_t. \label{diff_begin}\end{equation}

From (\ref{Ghost_eq}) and (\ref{diff_begin}) we have
\begin{equation}
\abs{{{B}}_t(\bm{s}) - {{X}}^N_t(\bm{s})} \leq \sum_{i=1}^t D_{i,t-i}^N.
\label{diff_1}
\end{equation}

Before proceeding with further analysis, we assume two results (\ref{diff_2}) and (\ref{diff_4}), which we will show later. 

For some constant $\hat{\beta} > 0$,
 \begin{equation}\sum_{i=1}^t \Exp{D_{i,t-i}^N} \le \frac{\hat{\beta}}{N} \bm{1}^T \sum_{i=0}^{t} {K}^{t-i} \Exp{(A_{i-1}^B)^2 }\bm{1}, 
 \label{diff_2}
 \end{equation}
 
 where $A_i^B$ is total number of affected individuals in branching process at time $i$ and $\bm{1} \in {\Re^+}^{\eta}$ is a vector with each entry equal to 1. 
 
 For some constant $\tilde{c}$,
 \begin{equation}\Exp{\lrp{A^B_i}^2} \leq \tilde{c} \rho^{2i}. \label{diff_4}
 \end{equation}
 
Assuming (\ref{diff_2}) to be true, from (\ref{diff_1}) and (\ref{diff_2}) we have
 
 \begin{equation}\Exp{\abs{{{B}}_t(\bm{s}) - {{X}}^N_t(\bm{s})}} \le \frac{\hat{\beta}}{N} \bm{1}^T \sum_{i=0}^{t} {K}^{t-i} \Exp{(A_{i-1}^B)^2 } \bm{1}.
 \label{diff_3}
 \end{equation}

 As $\lim_{t \to \infty} \frac{K^t}{\rho^t} = \bm{{u}}\bm{v}^T$, therefore there exists $K_3 \in {\Re^+}^{\eta\times\eta}$ such that for all $ t\ge 0 $
 \begin{equation}K^t \le {\rho}^t  K_3.\label{matrix_inequality_new}
 \end{equation}
 
 Assuming (\ref{diff_4}) to be true, from (\ref{diff_4}), (\ref{diff_3}) and (\ref{matrix_inequality_new}) we have,

\begin{equation*} \label{eq:Bound1}
      \begin{aligned}
        \Exp{\abs{{{B}}_t(\bm{s}) - {{X}}^N_t(\bm{s})}}   &\le \frac{\hat{\beta}}{N} \bm{1}^T \sum_{i=0}^{t} \rho^{t-i} K_3 \tilde{c} \rho^{2i-2} \bm{1}\\
        %&\le \frac{\hat{\beta}}{N} \bm{1}^T \sum_{i=0}^{t}    {K_3 \tilde{c}} \rho^{t+i-2} \bm{1}\\
                                          &\le  \frac{\hat{\beta}}{N}  \bm{1}^T    K_3 \tilde{c} \frac{\rho^{2t-1}}{(\rho -1)}  \bm{1} \\
                                          &= \tilde{e} \frac{\rho^{2t}}{N} , 
      \end{aligned}
    \end{equation*}
   
    where $\tilde{e} = \bigg(\frac{\hat{\beta}}{ \rho} \bm{1}^T    \frac{K_3 \tilde{c}}{(\rho -1)}  \bm{1}\bigg)$.




 
\textbf{Proof of (\ref{diff_2}):} Recall that ghost individuals $G_t^N$ is the number of uncoupled exposed individuals born from the coupled individuals in branching process when compared to epidemic process at time $t$. These extra individuals in branching process are born whenever a coupled infectious individual in epidemic process makes contact with already affected individual.
Therefore, $G^N_t$ is equal to the number of contacts made to already affected individuals in epidemic process in one time step between time $t-1$ and $t$. Let ${Y}_t^N$ be the total number of community contacts in one time step between time $t-1$ and $t$. For a given contact, probability of hitting an already affected individual is less then $\psi_{max} \frac{A^N_{t-1}+{Y}_t^N}{N}$, where $\psi_{max} = \max_{a\in \mathcal A, \tilde{a}\in\mathcal A}\{\psi_{a,\tilde{a}}\}$. Therefore, we have,

\begin{equation*}
    \begin{aligned}
    \Exp{{G}^N_t} &= \Exp{\Exp{{G}^N_t | \mathcal{F}_{t-1}} } \\
    & \leq \Exp{\Exp{\psi_{max} \frac{(A_{t-1}^N+{Y}_t^N)}{N} {Y}_t^N |\mathcal{F}_{t-1}}} \\
    & \leq \Exp{\Exp{\psi_{max} \frac{A_{t-1}^N{Y}_t^N}{N} |\mathcal{F}_{t-1}}} + \Exp{\Exp{ \psi_{max} \frac{({Y}_t^N)^2}{N} |\mathcal{F}_{t-1}}}. 
    \end{aligned}
\end{equation*}

To upper bound ${Y}_t^N$, we assume that all the affected individuals in epidemic process at time $t-1$ are infectious and that community hits by each infectious individual is Poisson distributed with $\beta_{max} = \max_{a\in \mathcal A}\{\beta_{a}\}$. Then,
\[\Exp{{G}^N_t} \leq \psi_{max}\Exp{ \beta_{max}\frac{(A_{t-1}^N)^2}{N} } + \psi_{max} \Exp{\frac{\beta_{max} A_{t-1}^N+\beta_{max}^2 (A_{t-1}^N)^2}{N} }. \]



Since $A_{t-1}^N\geq 1$, and setting $\hat{\beta}=\psi_{max}(\beta_{max}^2+2\beta_{max})$, we have
\[\Exp{{G}^N_t} \leq \hat{\beta} \Exp{ \frac{(A_{t-1}^N)^2}{N} }. \]

 To upper bound the $\Exp{D_{i,t-i}^N}$, we assume that each state $\bm{s} \in \mathcal S \setminus  \mathcal U $ has $G_i^N$ ghost individuals born at time $i$. As ghost individuals $G_i^N$ born at time $i$ have descendants according to the same branching process dynamics, 
\begin{equation*} \Exp{D_{i,t-i}^N} \le \frac{\hat{\beta}}{N} \bm{1}^T  {K}^{t-i} \Exp{(A_{i-1}^N)^2 }\bm{1}. 
 \end{equation*}

Taking summation on both sides we have,
 \begin{equation}\sum_{i=1}^t \Exp{D_{i,t-i}^N} \le \frac{\hat{\beta}}{N} \bm{1}^T \sum_{i=1}^{t} {K}^{t-i} \Exp{(A_{i-1}^N)^2 }\bm{1}. 
 \label{diff_end}
 \end{equation}


From (\ref{total_affected_ineq}) and (\ref{diff_end}), we have
 \begin{equation*}\sum_{i=1}^t \Exp{D_{i,t-i}^N} \le \frac{\hat{\beta}}{N} \bm{1}^T \sum_{i=1}^{t} {K}^{t-i} \Exp{(A_{i-1}^B)^2 }\bm{1}. 
 \end{equation*}



% Assume that, for any state $\bm{s} \in \mathcal A \times \mathcal D \setminus \{u\}$ and some constant $\hat{\beta}>0$
% \[\Exp{\abs{{{B}}_t(\bm{s}) - {{X}}^N_t(\bm{s})}}  \le \bigg(\frac{\hat{\beta}}{N} \bm{1}^T \sum_{i=0}^{t} {K}^{t-i} \Exp{(A_{i-1}^B)^2 } \bm{1}\bigg)\]
 
% Also assume that, \[\Exp{\lrp{A^B_i}^2} \leq \tilde{c} \rho^{2i} \]
 
 
% \begin{equation*} \label{eq:Bound1}
%      \begin{aligned}
%        \Exp{\abs{\bm{{B}}_t - \bm{{X}}^N_t}}  &\le \bigg(\frac{\hat{\beta}}{N} \bm{1}^T \sum_{i=0}^{t} {K}^{t-i} \Exp{(A_{i-1}^B)^2 } \bm{1}\bigg) \bm 1\\  &\le \bigg(\frac{\hat{\beta}}{N} \bm{1}^T \sum_{i=0}^{t} \rho^{t-i} K_\delta c_{\delta} \rho^{2i-2} \bm{1}\bigg) \bm 1\\
%        &\le \bigg(\frac{\hat{\beta}}{N} \bm{1}^T \sum_{i=0}^{t}    {K_\delta c_{\delta}} \rho^{t+i-2} \bm{1}\bigg) \bm 1\\
%                                          &\le \frac{\rho^{2t}}{N} \bigg(\frac{\hat{\beta}}{ \rho} \bm{1}^T    \frac{K_\delta c_{\delta}}{(\rho -1)}  \bm{1}\bigg) \bm 1\\
 %                                         &\le e_\delta \frac{\rho^{2t}}{N} \bm 1 , 
 %     \end{aligned}
 %   \end{equation*}
   
  %  where $e_\delta = \bigg(\frac{\hat{\beta}}{ \rho} \bm{1}^T    \frac{K_\delta c_{\delta}}{(\rho -1)}  \bm{1}\bigg)$
 




%\begin{itemize}
%    \item Step 1 : We prove that $\Exp{\abs{B_t(\bm{s})-X_t^N(\bm{s})}}$ for any state $\bm{s}$ of the epidemic process and branching process is upper bounded by 
%$\frac{ \hat{\beta}}{N} \bm{1}^T \sum_{i=1}^{t}    {K}^{t-i} \Exp{(A_{i-1}^B)^2 } \bm{1}$
%for some constant $\hat{\beta}$
% \item Step 2 : We prove that, $\Exp{\lrp{A^B_i}^2} \leq c_\delta \rho^{2i}$ where $ c_\delta $ is a constant
% \item Step 3 : Combine results from Step 1 and Step 2
%\end{itemize}
%\begin{itemize}
%    \item \textbf{STEP 1}


%\item \textbf{STEP 2}

\noindent \textbf{Proof of (\ref{diff_4}):} Recall that $A^B_i=\sum_{\bm{s}\in \mathcal {S} \setminus \mathcal U}B_i(\bm{s})$. Then, $A^B_i=\bm{1}^T \bm{B}_i= \bm{{B}}_i^T \bm{1}$. Hence,
    \begin{equation} 
      \begin{aligned}
        \Exp{\lrp{A^B_i}^2} &\leq \Exp{(\bm{1}^T {B}_i)^2 } \\
                            &=  \bm{1}^T \Exp{\bm{{B}}_i \bm{{B}}_i^T} \bm{1}.
     \label{affected_square_bound}
      \end{aligned}
    \end{equation}
    
       %Let, $\bm{e}_{\bm{s}} \in \mathbb{{R^+}^\eta}$ be a $\eta$ dimension vector whose component corresponding to type $\bm{s}$ is 1 and other components are 0. 
       Let $Var(\bm{B}_1)\in {\Re^+}^{\eta\times\eta}$, be a matrix with $(\bm{s},\bm{q})$ entry equalling $\Exp{B_1(\bm{s})B_1(\bm{q})}-\Exp{B_1(\bm{s})}\Exp{B_1(\bm{q})}$, $V_{\bm{s}}$ denote $Var(\bm{B}_1|\bm{B}_0=\bm{e}_{\bm{s}})$ and $C_0$ denote the matrix with $(\bm{s},\bm{q})$ entry set to $\Exp{B_0(\bm{s})B_0(\bm{q})}$. From (Chap. 2, pg 37, \cite{branching_process_notes_2}) we have,

\begin{equation}
      \begin{aligned}
      \Exp{\bm{{B}}_{i} \bm{{B}}_{i}^T} & = ({K}^i)^T C_0 {K}^i +  &  \sum_{j=1}^{i} ({K}^{i-j})^T \Big(\sum_{\bm{s}\in \mathcal {S} \setminus \mathcal U}V_{\bm{s}}\Exp{\bm{{B}}_{j-1}(\bm{s})}\Big) {K}^{i-j}. 
      \end{aligned}
      \label{bb_eq}
    \end{equation}
 
Furthermore, for our branching process, there exists a constant $v>0$ such that $V_{\bm{s}} \le v {K}$ for all $ \bm{s}\in \mathcal {S} \setminus \mathcal U$. Using this in (\ref{bb_eq}) we observe that
\begin{equation}
      \begin{aligned}
      \Exp{\bm{{B}}_{i} \bm{{B}}^T_{i}} & = ({K}^i)^T C_0 {K}^i +  &  v \sum_{j=1}^{i} ({K}^{i-j})^T \Big(\sum_{\bm{s}\in \mathcal {S} \setminus \mathcal {U}}\Exp{{B}_{j-1}(\bm{s})}\Big) {K}^{i-j+1}.
      \label{matrix_multiply_bound} 
      \end{aligned}
    \end{equation}
 


From (\ref{branching_basic_eq}) and (\ref{matrix_inequality_new}) we have  \begin{equation}
  \bm{B}_i \leq \rho^i (K_3)^T \bm{B}_0.
  \label{branching_basic_ineq}
\end{equation}.  

From (\ref{matrix_multiply_bound}) and (\ref{branching_basic_ineq}) we have


\begin{equation*}
    \begin{aligned}
    \Exp{\bm{{B}}_{i} \bm{{B}}^T_{i}} & \le \rho^{2i} (K_3)^T C_0 K_3 + v \tilde{d} \sum_{j=1}^{i} \rho^{2i-j} (K_3)^T  K_3\\
    & \le  \rho^{2i} \bigg((K_3)^T C_0 K_3 + \frac{\rho v \tilde{d}}{\rho-1}  (K_3)^T  K_3 \bigg)\\
    & = \rho^{2i} \tilde{K}_3,
    \label{eq:BBT3} 
    \end{aligned}
\end{equation*}
where, $\tilde{d} := \sum_{\bm{s}\in\mathcal S \setminus  \mathcal U} \Big(\Exp{{B}_0^T} K_3\Big)(\bm{s})$ and $\tilde{K}_3 := \bigg((K_3)^T C_0 K_3 + \frac{\rho v \tilde{d}}{\rho-1}  (K_3)^T  K_3\bigg)$. 
 
 

 Using above bound in (\ref{affected_square_bound}) and setting $\tilde{c}=\bm{1}^T \tilde{K}_3 \bm{1}$, we get
 
 \begin{equation*}
    \begin{aligned}\Exp{\lrp{A^B_i}^2} \leq \tilde{c} \rho^{2i}. 
    \end{aligned}
\end{equation*}
 
 

%\item \textbf{STEP 3}

%Using bound \ref{affected_square_bound_final} in result from step 1, we get

% \begin{equation*} \label{eq:Bound1}
%      \begin{aligned}
%        \Exp{\abs{\bm{{B}}_t - \bm{{X}}^N_t}}  &\le \bigg(\frac{\hat{\beta}}{N} \bm{1}^T \sum_{i=0}^{t} {K}^{t-i} \Exp{(A_{i-1}^B)^2 } \bm{1}\bigg) \bm 1\\  &\le \bigg(\frac{\hat{\beta}}{N} \bm{1}^T \sum_{i=0}^{t} \rho^{t-i} K_\delta c_{\delta} \rho^{2i-2} \bm{1}\bigg) \bm 1\\
%        &\le \bigg(\frac{\hat{\beta}}{N} \bm{1}^T \sum_{i=0}^{t}    {K_\delta c_{\delta}} \rho^{t+i-2} \bm{1}\bigg) \bm 1\\
%                                          &\le \frac{\rho^{2t}}{N} \bigg(\frac{\hat{\beta}}{ \rho} \bm{1}^T    \frac{K_\delta c_{\delta}}{(\rho -1)}  \bm{1}\bigg) \bm 1\\
 %                                         &\le e_\delta \frac{\rho^{2t}}{N} \bm 1 , 
%      \end{aligned}
%    \end{equation*}
   
%    where $e_\delta = \bigg(\frac{\hat{\beta}}{ \rho} \bm{1}^T    \frac{K_\delta c_{\delta}}{(\rho -1)}  \bm{1}\bigg)$
 


%\end{itemize}








\subsubsection{Proof for Theorem \ref{e_b_close}}

  For $\zeta > 0$, we have

    \[
      \begin{aligned} 
      \mathbb{P}\lrp{ \sup\limits_{t\in[0,\log_\rho (N^*/\sqrt{N})]} \abs{{{X}}^N_t(\bm{s}) - {{B}}_t(\bm{s})} \ge \zeta }  &\le \frac{1}{\zeta} \Exp{\sup\limits_{t\in[0,\log_\rho (N^*/\sqrt{N})]} \abs{{{X}}^N_t(\bm{s}) - {{B}}_t(\bm{s})}}\\
          & \le  \frac{1}{\zeta} \sum\limits_{t=0}^{\log_\rho (N^*/\sqrt{N})} ~\Exp{\abs{{{X}}^N_t(\bm{s}) - {{B}}_t(\bm{s})}}, 
      \end{aligned}
    \]
    where the first inequality follows from the Markov's inequality, and the second follows from observing that maximum of non-negative random variables is bounded by their sum. Similarly,  

    \[ 
      \begin{aligned}
        \mathbb{P}\lrp{ \sup\limits_{t \in[0, \log_\rho N^*]} \abs{ \frac{{{X}}^N_t(\bm{s})}{\rho^t} - \frac{{{B}}_t(\bm{s})}{\rho^t} } \ge \zeta } &\le \frac{1}{\zeta} \Exp{\sup\limits_{t\in[0,\log_\rho N^*]} \abs{\frac{{{X}}^N_t(\bm{s})}{\rho^t} - \frac{{{B}}_t(\bm{s})}{\rho^t}}} \\
                &\le \frac{1}{\zeta} \sum_{t=0}^{\log_\rho N^*} ~\Exp{\abs{ \frac{{{X}}^N_t(\bm{s})}{\rho^t} - \frac{{{B}}_t(\bm{s})}{\rho^t}}}.
      \end{aligned}
    \]

    Thus, it is sufficient to show that

    \[ \lim\limits_{N\rightarrow\infty} ~ \lrset{  ~ \sum_{t=0}^{\log_\rho (N^*/\sqrt{N})} ~\Exp{\abs{{{X}}^N_t(\bm{s}) - {{B}}_t(\bm{s})}}}  = 0,\]

    and 
    \[ \lim\limits_{N\rightarrow\infty} ~ \lrset {  ~ \sum_{t=0}^{\log_\rho N^*} ~\Exp{\abs{ \frac{{{X}}^N_t(\bm{s})}{\rho^t} - \frac{{{B}}_t(\bm{s})}{\rho^t}}}} = 0. \]



    From Lemma \ref{bounding_diff_lemma_final}, we have,

    \begin{equation}
      \begin{aligned}
        \Exp{\abs{{{X}}_t^N(\bm{s}) - {{B}}_t(\bm{s})}}    &\le \tilde{e} \frac{\rho^{2t}}{N} , 
        \label{difference_bound}
      \end{aligned}
    \end{equation}
    
    Taking sum from $t =0$ to  $t=\log_\rho (N^*/\sqrt{N})$ above, we observe that
    
     \[ \sum\limits_{i=0}^{\log_\rho (N^*/\sqrt{N})} ~\Exp{\abs{{{X}}^N_i(\bm{s}) - {{B}}_i(\bm{s})}} \leq  \frac{\rho^{2 \log_\rho (N^*/\sqrt{N})}}{N}\frac{\tilde{e}}{\rho^2-1},\]
    
   
    
    
   recall that $N^*=N/\log{N}$. Therefore,
    \[ \lim\limits_{N\rightarrow \infty} ~ \Exp{ \sup\limits_{t\in [0, \log_\rho (N^*/\sqrt{N})]} \abs{{{X}}_t^N(\bm{s}) - {{B}}_t(\bm{s})} } = 0 . \]

    
    
    
    
    
    
    Using (\ref{difference_bound}) to bound $\sum_{i=0}^{t}\frac{1}{\rho^i}\Big(  \mathbb{E}(\abs{{{X}}_{i}^N(\bm{s})-{{B}}_{i}(\bm{s})})\Big)$  we have,

    \begin{equation*}
    \begin{aligned}
    \sum_{i=0}^{t}\frac{1}{\rho^i}\Big(\  \mathbb{E}(\abs{{X}_{i}^N(\bm{s})-{B}_{i}(\bm{s})})\Big) &
     \leq  \frac{\rho^{t}}{N}  \frac{\tilde{e}}{\rho-1} .
    \label{eq:bound6}
   \end{aligned}
\end{equation*}
    
    Setting $t=\log_\rho N^*$, we observe that
    \[ \sum\limits_{i=0}^{t} ~\Exp{\abs{\frac{{X}^N_i(\bm{s})}{\rho^i} - \frac{{B}_i(\bm{s})}{\rho^i}}}  \leq \frac{1}{\log N}   \frac{\tilde{e}}{\rho-1}.\]
    
   Thus,
    \[ \lim\limits_{N\rightarrow\infty}  ~ \Exp{\sup\limits_{t\in [0,\log_\rho N^*]} \abs{\frac{{X}^N_t(\bm{s})}{\rho^t} - \frac{{B}_t(\bm{s})}{\rho^t}}} = 0. \]





















    












  




  \subsubsection{Proof of Theorem \ref{deterministic_theorem}} \label{app_proof_DetEv}
  
  We will prove Theorem \ref{deterministic_theorem} through induction. For the ease of notation we hide $W$ in representing the mean field process $\bm{\bar{\mu}}_t(W)$ and $\bar{c}_t(W,a)$ in the proof. Assumption \ref{deterministic assumption} forms the base case for induction. Assume that  $\bm{\mu}^N_{t} \xrightarrow{\text{P}} \bm{\bar{\mu}}_{t} $. From the definition we know that for all $\bm{s}=({a},{d}) \in \mathcal S$,
  

 
 % We will prove the deterministic phase results through induction on time $t$. 
  
  
  
  


 %it suffices to show that $\mu^N_t \xrightarrow{\text{D}} \bar{\mu}_{t+1} $ (since $\bar{\mu}$ is constant for a given $W$). Assumption $\mu^N_0 \xrightarrow{\text{P}} \bar{\mu}_0 $ forms the base-case for induction. Let us assume that for $1\le l \le t$, $\mu^N_{l} \xrightarrow{P} \bar{\mu}_{l}$. 

%For $\nu\in\mathcal P(\mathcal S)$ and $f\in C_B(\mathcal S) $, let 
%\[ <f, \nu> ~ := ~\sum\limits_{\bm{s}\in\mathcal S} f(\bm{s}) \nu(\bm{s}).\]

%Then, to establish the convergence in distribution at $l=t$, it suffices to show 
%\begin{equation}
%  \lim\limits_{N\rightarrow\infty}~ <f,\mu^N_t> ~ = ~<f, \bar{\mu}_t>,\label{req1}
%\end{equation}
%(see, \cite{giesecke}). 

\[\mu^N_{t+1}(\bm{s}) = \frac{1}{N} \sum\limits_{n=1}^N \mathbbm{1}\left(S^{t+1}_n = \bm{s}\right).\]
Recall that

 $$\mathbb{P}\left( S^{t+1}_n = \bm{s} | \mathcal F_{t} \right) = {h}(S^{t}_n, \bm{s},  c^N_{t}({a})),$$

Define $\mathcal M^N_{t+1}(\bm{s}) $   to be 
\begin{equation*}
  \frac{1}{N} \sum\limits_{n=1}^N \mathbbm{1}\left(S^{t+1}_n = \bm{s}\right) - {h}(S^{t}_n, \bm{s}, c^N_{t}({a})).
\end{equation*}

Clearly,
\begin{equation}\label{eq:errorVar} \mu^N_{t+1}(\bm{s})= \sum\limits_{\bm{s'}\in\mathcal S} ~ \left[  {h}(\bm{s'},\bm{s},c^N_{t}({a})) \mu^N_{t}(\bm{s'})  ~+~ \mathcal M^N_{t+1}(\bm{s})\right].\end{equation}


Observe that $\mathcal M^N_{t+1}(\bm{s})$ is a sequence of $0$-mean random variables whose variance converges to $0$ as $N\rightarrow \infty$. Therefore

\[\mathcal M^N_{t+1}(\bm{s}) \xrightarrow{P} 0. \]



Assuming  that as $N \to \infty$, 
%and as ($|\mathcal S| < \infty$),

\begin{equation}
{h}(\bm{s'},\bm{s},c^N_{t}({a}))\mu^N_{t}(\bm{s'}) \xrightarrow{P} {h}(\bm{s'},\bm{s},\bar{c}_{t}({a})) \bar{\mu}_{t}(\bm{s'}),
\label{deterministic_lemma}
\end{equation}
  we get
\[ {\mu}_{t+1}^N(\bm{s}) \xrightarrow{P} \sum\limits_{\bm{s'}\in \mathcal S} {h}(\bm{s'},\bm{s},\bar{c}_{t}({a}))\bar{\mu}_{t}(\bm{s'})  ~ =: ~ \bar{\mu}_{t+1}(\bm{s}). \]
 

% For $\bm{s}\in\mathcal S$, $\mathcal M^N_{t,f}(\bm{s}) \xrightarrow{P}0$ as $N\rightarrow \infty$.



To see (\ref{deterministic_lemma}), observe that  ${h}(\bm{s'},\bm{s},c^N_{t-1}({a})) \mu^N_{t}(\bm{s'})$, can be re-written as
  \begin{align*}
  &{h}(\bm{s'},\bm{s},\bar{c}_{t}({a})) \mu^N_{t}(\bm{s'})+\\&  ({h}(\bm{s'},\bm{s},c^N_{t}({a}))-{h}(\bm{s'},\bm{s},\bar{c}_{t}({a}))) \mu^N_{t}(\bm{s'}).
  \end{align*}
  Taking limit as $N\rightarrow \infty$ for the first term above, we get
  $${h}(\bm{s'},\bm{s},\bar{c}_{t}({a})) \mu^N_{t}(\bm{s'})  \xrightarrow{P} {h}(\bm{s},\bm{s'},\bar{c}_{t}({a})) \bar{\mu}_{t}(\bm{s'}),$$
  since $\mu^N_{t}(\bm{s})$ converges to $\bar{\mu}_{t}$ in probability, by induction hypothesis. Moreover, the second term equals $0$. To see this, observe that the second term above can be bounded by 
  \begin{equation} \label{eqn:conv0}
      \abs{{h}(\bm{s'},\bm{s},c^N_{t}({a}))-{h}(\bm{s'},\bm{s},\bar{c}_{t}({a}))}.
  \end{equation}
  
  From (\ref{cnt}), we can see that $c^N_{t}({a})$ is a continuous function of $\mu^N_{t}$ and $\mu^N_{t} \xrightarrow{P} \bar{\mu}_{t}$ by induction hypothesis. Therefore $$c^N_{t}({a})~\xrightarrow{P}  ~ \sum\limits_{\bm{q} \in \mathcal S} \bar{\mu}_{t}(\bm q)\mathbbm{1}(\text{state }\bm{q} \text{ is infectious}) \psi_{\tilde{a},{a}} \beta_{\tilde{a}}  = ~\bar{c}_{t}({a}). $$
  
  
 Observe that $c^N_{t}$ and $\bar{c}_{t}$ are bounded (by definition). Moreover, ${h}$ is uniformly continuous in its arguments since it is a continuous function defined on a compact set. Thus, taking limit as $N\rightarrow \infty$, (\ref{eqn:conv0}) converges to zero 
  since ${h}(\bm{s'},\bm{s},c^N_{t}({a})) \xrightarrow{P} {h}(\bm{s'},\bm{s},\bar{c}_t({a}))$ uniformly as $N\rightarrow\infty$.
 

  

\subsection{Bias in the smaller model: A heuristic justification}
\label{heuristic_initial}

In this section we compare infections in a large city  with the  appropriately scaled version of the smaller  city and show that the
smaller city underestimates the reported number of infections compared to the larger city. This bias  is more pronounced
when the initial number of infections are small, and becomes much less significant when this initial number infected  becomes large. 
As our analysis suggests, it is the initial variability in the infection process that causes this bias. 
 For simplicity of presentation, we use deterministic SIR model as used in Section \ref{branching_results}.

 %For the ease of notation, time step is 1. For this, let community transmission rate of an infected individual be $\beta$ (i.e. number of contacts made by an individual with other individuals in one time step is Poisson distributed with rate $\beta$) and an infected individual recovers at rate $r$. Consider a city with total population $mN$, where $m=1$ for smaller city $m=k$ for the larger city. Let number of infection and recovered people at time $t$ in this city be denoted by $i_{mN}(t)$, $r_{mN}(t)$. Therefore, for an SIR model we have :

First consider a city with total population $N$. Assume that there are  $i_0$ infected in the city at time 0. Let $i_{N}(t)$ and $r_{N}(t)$ be the number  infected and recovered at time $t$. Denote by $i^B(t)$ the number of infections in the associated  branching process at time $t$ when starting from 1 infection at time 0.


We know that for large $N$ till time $t_N = \log_{\rho} \frac{N^*}{i_0} $ (recall that $N^*=N/\log N$) the city evolves approximately as a branching process. Therefore, each starting infection gives rise to an  approximately independent tree of descendant infections till this time, and
\[i_{N}(t_N) = \sum_{j=1}^{i_0}i^B_j(t_N) + o(\rho^{t_N}),\]
where $i^B_j(t_N)$ for each $j\in [1,i_0]$ is distributed as an independent copy of $i^B(t)$. Recall from Example 3.1 that the proportion amongst the  infected and the recovered stabilises in the branching process as time increases. Hence, 

\[{r_{N}(t_N)} = \frac{r}{\beta-r}\sum_{j=1}^{i_0} {i^B_j(t_N)}+ o(\rho^{t_N}), \] 
and taking expectations, 
\[\Exp{i_{N}(t_N)}={i_0}\Exp{i^B(t_{N})} + o(\rho^{t_N}).\]

$\Exp{i_{N}(t_{N}+1)}$ can be estimated as follows,
\begin{equation}
 \begin{aligned}
\Exp{i_{N}(t_{N}+1)} &= \Exp{\beta \sum_{j=1}^{i_0}i^B_j(t_{N}) \bigg(1-\frac{1}{N}\big(\sum_{j=1}^{i_0}i^B_j(t_{N})+\sum_{j=1}^{i_0}r^B_j(t_{N})\big)\bigg)} + \Exp{(1-r) \sum_{j=1}^{i_0}i^B_j(t_{N})}+ o(\rho^{t_N})
\\&=\Exp{\beta \sum_{j=1}^{i_0}i^B_j(t_{N}) \bigg(1-\frac{1}{N}\big(\sum_{j=1}^{i_0}i^B_j(t_{N})+\frac{r}{\beta-r}\sum_{j=1}^{i_0}i^B_j(t_{N})\big)\bigg)}+ \frac{1}{N}(1-r) {i_0}\Exp{i^B(t_{N})}+ o(\rho^{t_N})
\\&=\Exp{\beta \sum_{j=1}^{i_0}i^B_j(t_{N}) \bigg(1-\frac{\beta}{N(\beta-r)}\sum_{j=1}^{i_0}i^B_j(t_{N})\bigg)} + \frac{1}{N}(1-r) {i_0}\Exp{i^B(t_{N})}+ o(\rho^{t_N})
\\&={(1+\beta-r) {i_0}\Exp{i^B(t_{N})} - \frac{\beta^2i_0}{N(\beta-r)}\Exp{[i^B(t_{N})]^2} - \frac{\beta^2i_0(i_0-1)}{N(\beta-r)}\Big[\Exp{i^B(t_{N})}}\Big]^2 + o(\rho^{t_N})
\\&={(1+\beta-r) {i_0}\Exp{i^B(t_{N})} - \frac{\beta^2i_0}{N(\beta-r)}\text{Var}(i^B(t_{N})) - \frac{\beta^2i_0^2}{N(\beta-r)}\Big[\Exp{i^B(t_{N})}}\Big]^2+ o(\rho^{t_N}).
 \label{smaller_city_next_time_step}
  \end{aligned} 
\end{equation}
  
 
 
 Now consider a larger city with total population $kN$ for $k>1$. At time 0, we proportionately increase the initial infections to 
  $ki_0$. Let $i_{kN}(t)$ and $ r_{kN}(t)$ be the number of infections and recovered at time $t$ in this larger city.


Again,  till time $t_{kN} = \log_{\rho} \frac{kN^*}{ki_0}= \log_{\rho} \frac{N^*}{i_0} = t_N$, the city evolves as a branching process. Therefore, each starting infection has an approximately independent infection tree till this time, and
\[i_{kN}(t_N) = \sum_{j=1}^{ki_0}i^B_j(t_N) + o(\rho^{t_N}),\]

where again each $i^B_j(t_N)$ is distributed as an independent copy of $i^B(t)$.  Again,

\[{r_{kN}(t_N)} = \frac{r}{\beta-r}\sum_{j=1}^{ki_0}{i^B_j(t_N)} + o(\rho^{t_N}).\] 

Following similar steps as earlier to estimate $\Exp{i_{kN}(t_{N}+1)}$, we have



\begin{equation}
\Exp{i_{kN}(t_{N}+1)}={k (1+\beta-r) {i_0}\Exp{i^B(t_{N})} - \frac{\beta^2i_0}{N(\beta-r)} \text{Var}{(i^B(t_{N}))} - k \frac{\beta^2i_0^2}{N(\beta-r)}\Big[\Exp{i^B(t_{N})}}\Big]^2 + o(\rho^{t_N}).
\label{bigger_city_next_time_step}
\end{equation}

Scaling $k$ times the estimate of $i_N(t_N+1)$ in (\ref{smaller_city_next_time_step}), we have

\begin{equation}  
  k\Exp{i_{N}(t_{N}+1)}={k(1+\beta-r) {i_0}\Exp{i^B(t_{N})} - k\frac{\beta^2i_0}{N(\beta-r)}\text{Var}(i^B(t_{N})) - k\frac{\beta^2i_0^2}{N(\beta-r)}\Big[\Exp{i^B(t_{N})}}\Big]^2+ o(\rho^{t_N}).
 \label{smaller_city_next_time_step_2}
\end{equation}

Observe that the second term $[\frac{\beta^2i_0}{N(\beta-r)} \text{Var}{(i^B(t_{N}))}]$ on right-hand side of (\ref{bigger_city_next_time_step}) is smaller than second term $[k\frac{\beta^2i_0}{N(\beta-r)}\text{Var}(i^B(t_{N}))]$ on  right-hand side of (\ref{smaller_city_next_time_step_2}). Other terms are equal in  (\ref{bigger_city_next_time_step}) and  (\ref{smaller_city_next_time_step_2}). This suggests that $\Exp{i_{kN}(t_{N}+1)} > k \Exp{i_{N}(t_{N}+1)} $. Thus a large city with $kN$ population should  have more infections at time $t_N+1$ as compared to the scaled smaller city with population $N$. Observe  that the variance term  in the output (\ref{smaller_city_next_time_step_2}) of smaller model induces an error when we estimate  the larger model by scaling the output of smaller model. Inductively, this error between the larger model and the scaled smaller model can be seen to hold till O($log_\rho N$) time. This is also evident in Figure \ref{128_infections}.

Also observe that in the estimate of $k\Exp{i_{N}(t_{N}+1)}$ for smaller city, the ratio of the third term to the second term for smaller city is $ i_0 \frac{\Big[\Exp{i^B(t_{N})}\Big]^2}{\text{Var}(i^B(t_{N}))}$, that is the third term becomes dominant over the second term as $i_0$ increases. As the third term is the same for both the larger city and the scaled smaller city, therefore as $i_0$ (initial number of infections) increases, the difference between the number of infections in the larger city and the scaled smaller city becomes smaller,  suggesting that scaling the smaller city provides
a good approximation to the larger city when $i_0$ is large. This is also evident from Figure \ref{12800_infections}. 





\subsection{Input data for numerical experiments} \label{Numerical Parameters Section}
In this section for completeness, we summarise the city statistics and disease related parameters used in our numerical experiments. 
This data is similar to that reported in \cite{City_Simulator_IISc_TIFR_2020} where it is more fully justified.

Recall that for the numerical experiments, we first create a synthetic city that closely models the actual Mumbai. A synthetic model is set to match the numbers employed, numbers in schools, commute distances, etc in Mumbai.  
%Larger model is with 12.8 million population and smaller model is with 1 million population. 
Tables \ref{Household size} and \ref{School size} show the household size distribution and school size distribution in the model.  Fraction of working population is set to 40.33\%. Workplace size distribution can be seen in \cite{City_Simulator_IISc_TIFR_2020}. 

For Figures \ref{12800_infections} to \ref{fatal_hq_40}, we consider \textbf{one community space} and the whole population is considered to be living in non-slums.  For Figure \ref{shift_scale_restart_real_intervention},
we consider \textbf{48 community spaces}, to model the 24 administrative wards in Mumbai further divided into slums and non-slums.
Mumbai slums are densely crowded. This  leads to difficulties in maintaining social distancing  and increases the transmission rate between the infected and the uninfected. We account for this by selecting a higher community transmission rate in the slums (2 times the non-slums). 
This leads to slum and non-slum prevalence that closely matches the seroprevalence data observed in July 2020 (see \cite{malani2020serosurvey}).
Table \ref{non slum age} summarises the non-slum population age distribution  and Table \ref{Slum age} summarises the non-slum population age distribution. 

 

 
 
 Transmission rates are set as follows : $\beta_h=1.227$, $\beta_w=0.919$, $\beta_s=1.82$, $\beta_c=0.233$. See, Table \ref{Disease Progression} and Figure \ref{fig:diseaseProgression} for details of the disease progression.  Symptomatic patients are assumed to be more infectious during the symptomatic period than during the pre-symptomatic infective stage (1.5 times more infectious in our model).
   
   Table \ref{interventions} summarizes the details of interventions modelled in the simulator.
   In Figures \ref{12800_infections} to \ref{fatal_no_int}, no interventions were implemented. In Figures \ref{shift_scale_restart_1} to \ref{fatal_hq_40}, intervention was home quarantine from day 40. For Figure \ref{shift_scale_restart_real_intervention}, 
   interventions were introduced based on the actual interventions as happened in the city. Specifically,
   no intervention for first 33 days, i.e. till 20 March, 2020 (Simulation starts on 13th Feb, 2020). Lockdown from March 20 to May 17. Masks are active from 9 April, 2020 (from day 53). Rules for higher social distancing of elderly are enforced from May 1, 2020 (from day 75). 
   Schools are closed throughout the pandemic period except in the earlier no intervention period. 
   Other interventions such as home quarantine and  case isolation are also implemented post the lockdown. 
   Attendance schedules at workplaces after May 17 are set as follows : 5\% in May 18-31, 20\% in all of June , 33\% in all of July, and 50\% thereafter. 
   These numbers were selected keeping in mind the prevalent restrictions and observing the transportation and the Google mobility data. 
   Compliance levels are set at 60\% in non-slums and 40\% in slums. These appeared reasonable and these  matched the fatality data early on in the pandemic (till June 2020).
   
  
%  Similarly for Figure \ref{ssr_real_int_with_new_strain}, interventions were introduced based on the actual interventions as happened in the city. Specifically, no intervention for first 33 days, i.e. till 20 March, 2020 (Simulation starts on 13th Feb, 2020). Lockdown from March 20 to May 17. Masks are active from 9 April, 2020 (from day 53). Rules for higher social distancing of elderly are enforced from May 1, 2020 (from day 75).   Schools are closed throughout the pandemic period except in the earlier no intervention period.  Other interventions such as home quarantine and  case isolation are also implemented post the lockdown.    Attendance schedules at workplaces after May 17 are set as follows : 5\% attendance from May 18 to May 31st, 2020, 15\% attendance in June, 25\% in July, 33\% in August, 50\% from September 2020 to January, 2021, 65\% from February, 2021 and 20\% from 15 April - 31 May 2021.  To account for increased intermingling due to the Ganpati festival, from August 20 to September 1, we increase $\beta$ for community by 2/3, and we reduce compliance from 60\% in non-slums and 40\% in slums to 40\% in non-slums and 20\% in slums. Similarly for Diwali and Christmas festivities.   Compliance levels are set at 60\% in Non slums (NS) and 40\% in slums (S) before December 2020 and outside festivals (during festival periods compliance is 40\% NS and 20\%  S). These change to (50\%, 30\%) in  Dec 2020, (40\%, 20\%) in Jan 2021, (20\%, 10\%) in 1-18 February 2021 and (40\%, 20\%)  from Feb 19 to April 14, 2021.  During the lockdown 15 April - 31 May 2021, it is (60\%, 40\%). We assume that there existed a single variant that accounted for 2.5\% of all the infected population on Feb. 1 in our model. These were randomly chosen amongst all the infected on Feb 1. Further we assumed that this variant was 2.25 times more infectious compared to the original strain.
  
  
  
  
  \begin{table}
  \centering
    \small
    \caption{Non-slum population age distribution \label{non slum age}}
    \begin{tabular}{|c|c|}
      \hline
      {\bf Age (yrs)} & {\bf Fraction of population}\\
      \hline
      0-4 & 0.0757\\
      5-9 & 0.0825 \\ 
      10-14 & 0.0608\\
      15-19 & 0.0669 \\ 
      20-24 & 0.0705\\
      25-29 & 0.0692 \\ 
      30-34 & 0.0777\\
      35-39 & 0.0716 \\ 
      40-44 & 0.0762\\
      \hline
    \end{tabular}
    \quad
     \begin{tabular}{|c|c|}
      \hline
      {\bf Age (yrs)} & {\bf Fraction of population}\\
      \hline
      45-49 & 0.0664 \\ 
      50-54 & 0.0795\\
      55-59 & 0.0632 \\ 
      60-64 & 0.0560\\
      65-69 & 0.0380 \\ 
      70-74 & 0.0227\\
      75-79 & 0.0136 \\
      80+ & 0.0094\\
      \hline
    \end{tabular}
\end{table}


    
  
\begin{table}
  \centering
    \small
    \caption{Slum population age distribution \label{Slum age}}
    \begin{tabular}{|c|c|}
      \hline
      {\bf Age (yrs)} & {\bf Fraction of population}\\
      \hline
      0-4 & 0.0757\\
      5-9 & 0.0825 \\ 
      10-14 & 0.0875\\
      15-19 & 0.0963 \\ 
      20-24 & 0.0934\\
      25-29 & 0.0917 \\ 
      30-34 & 0.0921\\
      35-39 & 0.0849 \\ 
      40-44 & 0.0606\\
      \hline
    \end{tabular}
    \quad
     \begin{tabular}{|c|c|}
      \hline
      {\bf Age (yrs)} & {\bf Fraction of population}\\
      \hline
      45-49 & 0.0529 \\ 
      50-54 & 0.0632\\
      55-59 & 0.0503 \\ 
      60-64 & 0.0327\\
      65-69 & 0.0221 \\ 
      70-74 & 0.0073\\
      75-79 & 0.0044 \\
      80+ & 0.0021\\
      \hline
    \end{tabular}
\end{table}



\begin{table}
  \centering
    \small
    \caption{Household size distribution \label{Household size}}
    \begin{tabular}{|c|c|}
      \hline
      {\bf Household size} & {\bf Fraction of households}\\
      \hline
      1 & 0.0485\\
      2 & 0.1030 \\ 
      3 & 0.1715\\
      4 & 0.2589 \\ 
      5 & 0.1819\\
      \hline
    \end{tabular}
    \quad
     \begin{tabular}{|c|c|}
      \hline
      {\bf Household size} & {\bf Fraction of households}\\
      \hline
      6 & 0.1035 \\ 
      7-10 & 0.1165\\
      11-14 & 0.0126 \\ 
      15+ & 0.0035\\
      \hline
    \end{tabular}
\end{table}

\begin{table}
  \centering
    \small
    \caption{School size distribution \label{School size}}
    \begin{tabular}{|c|c|}
      \hline
      {\bf School size} & {\bf Fraction of schools}\\
      \hline
      0-100 & 0.0185\\
      100-200 & 0.1284 \\ 
      200-300 & 0.2315\\
      300-400 & 0.2315 \\ 
      400-500 & 0.1574\\
      \hline
    \end{tabular}
    \quad
     \begin{tabular}{|c|c|}
      \hline
      {\bf School size} & {\bf Fraction of schools}\\
      \hline
      500-600 & 0.0889 \\ 
      600-700 & 0.063\\
      700-800 & 0.0481 \\ 
      800-900 & 0.0408\\
      900+ & 0\\
      \hline
    \end{tabular}
\end{table}
    

  
   


\begin{table}
  \centering
    \small
    \caption{Disease progression parameters \label{Disease Progression}}
    \begin{tabular}{|c|c|}
      \hline
      {\bf Parameter description} & {\bf Values}\\
      \hline
      Incubation Period & Gamma distributed with shape 2 and scale 2.29\\
      Asymptomatic Period & Exponentially distributed with mean duration 0.5 of a day \\ Symptomatic Period &  Exponentially distributed with mean duration of 5 days \\ Hospitalisation Period & 8 days \\ Critical Period & 8 days \\
      \hline
    \end{tabular}
\end{table}


\begin{figure}
      \centering
      
 \includegraphics{Graphs/disease_progression.jpg}
\caption{ A simplified model of COVID-19 progression.}
\label{fig:diseaseProgression}
\end{figure}


     

\begin{table}
  \centering
    \small
    \caption{Interventions as implemented in the simulator \label{interventions}}
    \begin{tabular}{|c|c|}
      \hline
      {\bf Intervention} & {\bf Description}\\
      \hline
      No intervention & Business as usual\\
      Lockdown & For compliant households, household rates are doubled, no workplace \\ & interactions except for 25\% leakage (for essential services), community \\ & interactions reduce by 75\%.  For non-compliant households, workplace \\ & interactions only have a leakage of 25\%,  community interactions are \\ & unchanged, and household interactions increase by 50\%\\
	  Case Isolation & Compliant symptomatic individuals stay at home for 7 days, reducing  non- \\ & household contacts by 75\%. Household contacts remain unchanged.\\
	  Home Quarantine & Following identification of a symptomatic case in a compliant household, \\ & all household members remain at home for 14 days. Household  contact \\ & rates double,  contacts in the community reduce by 75\%\\
	  Social distancing of the elderly & All compliant individuals over 65 years of age reduce their community \\ & interactions by 75\%\\
	  Schools and colleges closed & Self explanatory\\
	  Masks & Reduce community transmission by 20\%\\
      \hline
    \end{tabular}
\end{table}










%Assuming that above is true till some time $t_{N,c}= log_{\rho} \frac{\epsilon N}{ci_0}$,  the process $i^N_{c,k}(t_{N,c}+t)$  

%Let $\mu_{mN}(t)$ be obtained through normalizing $i_{mN}(t)$ by $m$. Therefore,

%\[\mu_{mN}(t_N) = \frac{1}{m}i_{mN}(t_N) = \frac{1}{m}\sum_{j=1}^{mi_0}i^B_j(t_{N})\]

%\[\Exp{\mu_{mN}(t_N)}=\frac{1}{m}\Exp{\sum_{j=1}^{mi_0}i^B_j(t_{N})}=\frac{mi_0}{mN}\Exp{i^B(t_{N})}=\frac{i_0}{N}\Exp{i^B(t_{N})}\]



 %Below we show that at the next time step a bias is introduced between larger and smaller city.

%\[\Exp{\mu_{mN}(t_{N}+1)}=\frac{1}{m}\Exp{\beta \sum_{j=1}^{mi_0}i^B_j(t_{N}) \bigg(1-\frac{1}{mN}\big(\sum_{j=1}^{mi_0}i^B_j(t_{N})+\sum_{j=1}^{mi_0}r^B_j(t_{N})\big)\bigg)}\]

%\[\Exp{\mu_{mN}(t_{N}+1)}=\frac{1}{m}\Exp{\beta \sum_{j=1}^{mi_0}i^B_j(t_{N}) \bigg(1-\frac{1}{mN}\big(\sum_{j=1}^{mi_0}i^B_j(t_{N})+\frac{1}{\beta-1}\sum_{j=1}^{mi_0}i^B_j(t_{N})\big)\bigg)}\]




%\[\Exp{\mu_{mN}(t_{N}+1)}=\frac{1}{m}\Exp{\beta \sum_{j=1}^{mi_0}i^B_j(t_{N}) \bigg(1-\frac{\beta}{mN(\beta-1)}\sum_{j=1}^{mi_0}i^B_j(t_{N})\bigg)}\]

      
%\[\Exp{\mu_{mN}(t_{N}+1)}=\frac{1}{m}\bigg({\beta {mi_0}\Exp{i^B(t_{N})} - \frac{\beta^2mi_0}{mN(\beta-1)}\Exp{(i^B(t_{N}))^2} - \frac{\beta^2mi_0(mi_0-1)}{mN(\beta-1)}\Big(\Exp{i^B(t_{N})}}\Big)^2\bigg)\]



    
%\[\Exp{\mu_{mN}(t_{N}+1)}={\beta {i_0}\Exp{i^B(t_{N})} - \frac{\beta^2i_0}{mN(\beta-1)}\text{Var}(i^B(t_{N})) - \frac{\beta^2i_0^2}{N(\beta-1)}\Big(\Exp{i^B(t_{N})}}\Big)^2\]
  
 
  
  
  



 



\end{document}


